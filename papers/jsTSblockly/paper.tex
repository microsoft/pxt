\documentclass[sigplan,10pt]{acmart}
%\settopmatter{printfolios=false,printccs=false,printacmref=false}
\usepackage{graphicx}
\usepackage{listings}
\usepackage{enumitem}
\newcommand{\MC}{MakeCode\ }
\newcommand{\MCN}{MakeCode}
\newcommand{\CO}{CODAL\ }
\newcommand{\CON}{CODAL}
\newcommand{\COLN}{codal}
\newcommand{\UF}{UF2\ }
\newcommand{\UFN}{UF2}
\newcommand{\flameon}[1]{\emph{#1}}
\def\dbhref#1#2{URL}
%\def\dbhref#1#2{\href{#1}{#2}}
\def\dburl#1{URL}
%\def\dburl#1{\url{#1}}

\setlist[itemize]{leftmargin=*}
\lstset{ %
language=C++,                % choose the language of the code
basicstyle=\footnotesize,       % the size of the fonts that are used for the code
numbers=left,                   % where to put the line-numbers
numberstyle=\footnotesize,      % the size of the fonts that are used for the line-numbers
stepnumber=1,                   % the step between two line-numbers. If it is 1 each line will be numbered
numbersep=5pt,                  % how far the line-numbers are from the code
backgroundcolor=\color{white},  % choose the background color. You must add \usepackage{color}
showspaces=false,               % show spaces adding particular underscores
showstringspaces=false,         % underline spaces within strings
showtabs=false,                 % show tabs within strings adding particular underscores
frame=single,           % adds a frame around the code
tabsize=2,          % sets default tabsize to 2 spaces
captionpos=b,           % sets the caption-position to bottom
breaklines=true,        % sets automatic line breaking
breakatwhitespace=false,    % sets if automatic breaks should only happen at whitespace
escapeinside={\%*}{*)}          % if you want to add a comment within your code
}

%% For double-blind review submission, w/ CCS and ACM Reference
%\documentclass[sigplan,10pt,review,anonymous]{acmart}\settopmatter{printfolios=true}
%% For single-blind review submission, w/o CCS and ACM Reference (max submission space)
%\documentclass[sigplan,10pt,review]{acmart}\settopmatter{printfolios=true,printccs=false,printacmref=false}
%% For single-blind review submission, w/ CCS and ACM Reference
%\documentclass[sigplan,10pt,review]{acmart}\settopmatter{printfolios=true}
%% For final camera-ready submission, w/ required CCS and ACM Reference
%\documentclass[sigplan,10pt]{acmart}\settopmatter{}


%% Conference information
%% Supplied to authors by publisher for camera-ready submission;
%% use defaults for review submission.
% \acmConference[PL'17]{ACM SIGPLAN Conference on Programming Languages}{January 01--03, 2017}{New York, NY, USA}
% \acmYear{2017}
% \acmISBN{} % \acmISBN{978-x-xxxx-xxxx-x/YY/MM}
% \acmDOI{} % \acmDOI{10.1145/nnnnnnn.nnnnnnn}
\startPage{1}

%% Copyright information
%% Supplied to authors (based on authors' rights management selection;
%% see authors.acm.org) by publisher for camera-ready submission;
%% use 'none' for review submission.
\setcopyright{none}
%\setcopyright{acmcopyright}
%\setcopyright{acmlicensed}
%\setcopyright{rightsretained}
%\copyrightyear{2017}           %% If different from \acmYear

%% Bibliography style
\bibliographystyle{ACM-Reference-Format}
%% Citation style
%\citestyle{acmauthoryear}  %% For author/year citations
%\citestyle{acmnumeric}     %% For numeric citations
%\setcitestyle{nosort}      %% With 'acmnumeric', to disable automatic
                            %% sorting of references within a single citation;
                            %% e.g., \cite{Smith99,Carpenter05,Baker12}
                            %% rendered as [14,5,2] rather than [2,5,14].
%\setcitesyle{nocompress}   %% With 'acmnumeric', to disable automatic
                            %% compression of sequential references within a
                            %% single citation;
                            %% e.g., \cite{Baker12,Baker14,Baker16}
                            %% rendered as [2,3,4] rather than [2-4].


%%%%%%%%%%%%%%%%%%%%%%%%%%%%%%%%%%%%%%%%%%%%%%%%%%%%%%%%%%%%%%%%%%%%%%
%% Note: Authors migrating a paper from traditional SIGPLAN
%% proceedings format to PACMPL format must update the
%% '\documentclass' and topmatter commands above; see
%% 'acmart-pacmpl-template.tex'.
%%%%%%%%%%%%%%%%%%%%%%%%%%%%%%%%%%%%%%%%%%%%%%%%%%%%%%%%%%%%%%%%%%%%%%


%% Some recommended packages.
\usepackage{booktabs}   %% For formal tables:
                        %% http://ctan.org/pkg/booktabs
\usepackage{subcaption} %% For complex figures with subfigures/subcaptions
                        %% http://ctan.org/pkg/subcaption

\usepackage{courier}

\begin{document}

%% Title information
\title{TypeScript: From JavaScript \\ to Blockly and Back}         %% [Short Title] is optional;
                                        %% when present, will be used in
                                        %% header instead of Full Title.
\subtitle{Microsoft MakeCode Team}                     %% \subtitle is optional


%% Author information
%% Contents and number of authors suppressed with 'anonymous'.
%% Each author should be introduced by \author, followed by
%% \authornote (optional), \orcid (optional), \affiliation, and
%% \email.
%% An author may have multiple affiliations and/or emails; repeat the
%% appropriate command.
%% Many elements are not rendered, but should be provided for metadata
%% extraction tools.

% %% Author with single affiliation.
% \author{James Devine}
% \affiliation{
%   \institution{Lancaster University, UK}            %% \institution is required
% }
% \email{james@devine.eu}  
% \author{Joe Finney}
% \affiliation{
%   \institution{Lancaster University, UK}            %% \institution is required
% }
% \email{j.finney@lancaster.ac.uk} 
% \author{Micha\l Moskal}
% \affiliation{
%   \institution{Microsoft, USA}            %% \institution is required
% }
% \email{mmoskal@microsoft.com} 
% \author{Peli de Halleux}
% \affiliation{
%   \institution{Microsoft, USA}            %% \institution is required
% }
% \email{jhalleux@microsoft.com} 
% \author{Thomas Ball}
% \affiliation{
%   \institution{Microsoft, USA}            %% \institution is required
% }
% \email{tball@microsoft.com} 
% \author{Steve Hodges}
% \affiliation{
%   \institution{Microsoft, UK}            %% \institution is required
% }
% \email{shodges@microsoft.com} 

%% Author with two affiliations and emails.
% \author{First2 Last2}
% \authornote{with author2 note}          %% \authornote is optional;
%                                         %% can be repeated if necessary
% \orcid{nnnn-nnnn-nnnn-nnnn}             %% \orcid is optional
% \affiliation{
%   \position{Position2a}
%   \department{Department2a}             %% \department is recommended
%   \institution{Institution2a}           %% \institution is required
%   \streetaddress{Street2a Address2a}
%   \city{City2a}
%   \state{State2a}
%   \postcode{Post-Code2a}
%   \country{Country2a}                   %% \country is recommended
% }
% \email{first2.last2@inst2a.com}         %% \email is recommended


%% Abstract
%% Note: \begin{abstract}...\end{abstract} environment must come
%% before \maketitle command
% \begin{abstract}
    % http://www.grandviewresearch.com/press-release/global-microcontroller-market
    Over the last decade, microcontrollers, the low-power low-cost workhorses of embedded systems, 
    are finding use in making and education. Furthermore, growth for microcontrollers is increasing due to
    demand for devices to monitor and control systems (e.g., Internet of Things, home automation).
    However, one generally needs a professional development environment 
    and substantial programming skills to develop applications for micrcontrollers. 
    
    We present a new open source platform for programming of microcontroller-based devices, with the goal
    of making it easy for \emph{anyone} to participate in creating with microcontrollers from 
    most \emph{anywhere} (a computer with a modern web browser and a USB port). 
    We evaluate the performance of the platform on devices ranging from the Arduino Uno
    to micro:bit and Adafruit Circuit Playground Express, popular boards used in making and creating
    around the world. We describe how the platform has been architected to make it easy to port
    to a wide range of microcontollers (\emph{anything}).

%    The platform supports visual block-based programming,
%    and JavaScript in a web app with editors, simulator and complete compile chain that compiles 
%    user code to binary in the browser. Static TypeScript, a subset of the TypeScript languge, 
%    mediates between worlds of JavaScript and the C++ runtime for the microcontroller, 
%    which is precompiled and cached in the web app. Both the JavaScript and C++ runtimes support
%    an event-based programming model with support for non-preemptive fibers provides a simple starting 
%    point for beginners, and progression to scenarios which benefit from concurrency. 
\end{abstract}

\begin{abstract}
    % http://www.grandviewresearch.com/press-release/global-microcontroller-market
    Over the last decade, microcontrollers, the low-power low-cost workhorses of embedded systems, 
    are finding use in making and education. Furthermore, growth for microcontrollers is increasing due to
    demand for devices to monitor and control systems (e.g., Internet of Things, home automation).
    However, one generally needs a professional development environment 
    and substantial programming skills to develop applications for micrcontrollers. 
    
    We present a new open source platform for programming of microcontroller-based devices, with the goal
    of making it easy for \emph{anyone} to participate in creating with microcontrollers from 
    most \emph{anywhere} (a computer with a modern web browser and a USB port). 
    We evaluate the performance of the platform on devices ranging from the Arduino Uno
    to micro:bit and Adafruit Circuit Playground Express, popular boards used in making and creating
    around the world. We describe how the platform has been architected to make it easy to port
    to a wide range of microcontollers (\emph{anything}).

%    The platform supports visual block-based programming,
%    and JavaScript in a web app with editors, simulator and complete compile chain that compiles 
%    user code to binary in the browser. Static TypeScript, a subset of the TypeScript languge, 
%    mediates between worlds of JavaScript and the C++ runtime for the microcontroller, 
%    which is precompiled and cached in the web app. Both the JavaScript and C++ runtimes support
%    an event-based programming model with support for non-preemptive fibers provides a simple starting 
%    point for beginners, and progression to scenarios which benefit from concurrency. 
\end{abstract}


%% Keywords
%% comma separated list
%\keywords{keyword1, keyword2, keyword3}  %% \keywords are mandatory in final camera-ready submission


%% \maketitle
%% Note: \maketitle command must come after title commands, author
%% commands, abstract environment, Computing Classification System
%% environment and commands, and keywords command.
\maketitle

\section{Introduction}
\label{sec:intro}

Over the last decade, microcontrollers, the workhorses of embedded systems, have become
central to efforts in making~\cite{dougherty2012maker} and education. For example, the Arduino project
(\url{www.arduino.cc})~\cite{buildingArduino2014},
started in 2003, created the Uno board using an 8-bit Atmel
AVR microcontroller. The Uno makes its microcontroller's I/O pins available via headers;
external hardware modules (shields) may be connected to these headers to extend
the Uno's capability. The Arduino ecosystem has grown tremendously in the past 15 years,
with the support of companies such as Adafruit Industries (\url{www.adafruit.com}) and
Sparkfun Electronics (\url{www.sparkfun.com}), who resell Arduino and make their
own Arduino-compatible boards. 

The Arduino platform has the following characteristics, common to many programming
environments for microcontrollers~\cite{XYZ}:
\begin{itemize}
\item it uses C/C++ as the starting programming language;
\item it loads code using 1980's era bootloader technology;
\item it encourages polling of sensors;
\item it lacks many interactive features of modern IDEs;
\end{itemize}
These characteristics make such systems non-trivial for beginners to work with, 
require the installation of OS-specific drivers/applications/toolchains,
and leads to poor programming practices.

The Java language (among others) held out the promise of a better way forward for 
programming microcontrollers, but XYZ.  
\flameon{we need more description of why current ways of programming microcontrollers 
make for a high barrier to entry; also need to take on Java head on here, as well as
RTOS and MicroPython.}

In contrast to the situation for programming microcontrollers, 
on the web we find many excellent environments for introductory programming.
Visual block editors such as Scratch (\url{https://scratch.mit.edu/})~\cite{ScratchCACM2009,BlocksBeyondCACM2017}
and Blockly (\url{https://developers.google.com/blockly/})~\cite{Blocky2015}
allow the creation of programs without the possibility of syntax errors.
The programming models associated with Scratch and Blockly generally are
event based, freeing the programmer from the need to poll.
HTML, CSS and JavaScript allow a complete programming experience to be delivered as an interactive
web app, including editing with intellisense, code execution and debugging~\cite{Monaco}. 

With a surge in the demand of microcontroller-based devices for education~\cite{XYZ}, 
there is a need to simplify the programming of such devices so that they suitable 
for novice users in restricted environments.
Therefore, we have created a new programming platform that bridges the worlds of 
the microcontroller and the web app. 

The major goals of the platform are to:
% TypeScript / Blocks + MakeCode
(1) make it simple to program microcontrollers in a higher-level language,
using nothing more than a web app;
% TypeScript and Blocks prevent users from making boo boos.
(2) provide a safe environment for users to develop programs for microcontrollers;
% simulator, auto completion...
(3) create a feature rich and extensible development environment that decreases time taken to program a microcontroller (time to awesome);
% UF2 is awesome
(4) allow a users' compiled program to be easily installed on a microcontroller;


The platform consists of a stack of four novel technologies, the subject of
this paper:
\begin{itemize}
\item \emph{\MC (\href{https://makecode.com}{makecode.com})}, a web app that supports both visual block programming and text programming,
via \emph{Static TypeScript}, with conversion between the two program representations (Section~\ref{sec:makecode});

\item \emph{Static TypeScript}, a statically-typed subset of TypeScript (\url{www.typescriptlang.org}),
a gradually-typed superset of JavaScript, for fast execution on low-memory devices, with
a simple model for linking against pre-compiled C++ (Section~\ref{sec:sts});

\item \emph{\CO (the Component-oriented Device Abstraction Layer)}, an event driven, multi-threaded, C++ runtime environment that bridges the semantic gap between higher-level languages and the hardware,
modelling each hardware component as a software component (Section~\ref{sec:codal});

\item \emph{USB Flashing Format} (UF2), a new file format designed for flashing microcontrollers 
over the Mass Storage Class protocol (USB pen drives); the format greatly speeds the installation of user
programs and is robust to differences in operating systems (Section~\ref{sec:uf2}).
\end{itemize}
The \MC web app is the entry point of the platform, and has in-browser execution via a device simulator, as well as compilation to machine code and linking against a
pre-compiled C++ runtime (\emph{\CON}). No C/C++ compiler is invoked to compile user code and the result of compilation is a binary file that is ``downloaded'' from the web app to the user's
computer and then flashed to the microcontroller (exposed as a USB pen drive) 
via a simple file copy operation,  with the aid of the \emph{UF2} file format and supporting firmware. 


These four advances enable beginners to get started programming microcontrollers from 
any modern web browser, and enable hardware vendors to innovate and safely add new 
components to the mix using Static TypeScript, leveraging its
foreign function interface to C++.
Once the web app has been loaded, all the above functionality works offline 
(i.e., if the host machine loses its connection
to the internet). 

\begin{table}[]
\centering
\begin{tabular}{|l|r|r|r|r|r|}
\hline
            &          &            & \bf{Word} &          &             \\
\bf{Device} & \bf{RAM} & \bf{Flash} & \bf{Size} & \bf{CPU} & \bf{Chip}   \\ \hline
Uno         & 2 kB     & 32 kB      & 8         & AVR      & ATmega328P  \\ \hline
micro:bit   & 16 kB    & 256 kB     & 32        & M0       & nRF52       \\ \hline
CPX         & 32 kB    & 256 kB     & 32        & M0       & SAMD21      \\ \hline
\end{tabular}
\caption{\label{table:devices}A subset of devices supported by the platform. 
CPX is Adafruit's Circuit Playground Express. M0 denotes Cortex-M0.}
\end{table}
      
Table~\ref{table:devices} lists three of the devices supported by our platform, ranging
from the highly resource-contrained Arduino Uno to the slightly less constrained space of
the micro:bit and Adafruit Circuit Playground Express (CPX).


\subsection{Running Example}

Figure~\ref{fig:example} shows a program in the Static
TypeScript subset that implements a simple ``stopwatch'' timer
for the micro:bit.
The program has three top-level statements:
the first initializes the global variable \emph{start} (line 1); the
second registers an event handler (a lambda function) to execute
each time button A of the micro:bit is pressed (line 3); the
third registers a lambda function to run forever on a fiber (line 18),
to animate the micro:bit's 5x5 LED screen whenever the timer is active. 

Note that this program is a JavaScript program, as there are no 
types mentioned explicitly. However, all the functions called in
this program are part of the runtime and are explicitly
typed.  As a result, the static type of every variable and expression
can be inferred by TypeScript's type inference.

The program also shows off the use of the non-preemptive concurrency
model supported by both \MC (for JavaScript) and \CO (for C++). 
The fiber running the forever statement executes the lambda inside a ``while (true)'' 
loop that yields (via a call to \emph{basic.pause}) after each call to the lambda.
This gives the button-press event handler a chance to execute
upon user input (in a separate fiber). Although the global variable \emph{start} is 
shared by the two fibers, there is no data race due to the non-preemptive 
scheduling model. 

\flameon{TODO: event queueing/execution model???}



\begin{figure}
\begin{lstlisting}
let start = 0

input.onButtonPressed(Button.A, () => {
  if (start == 0) {
    start = input.runningTime()
  } else {
    let d = input.runningTime() - start
    start = 0 
    basic.clearScreen()
    basic.pause(1000)
    basic.showString(d/1000 + "." + d%1000)
  }
})

basic.forever(() => {
  if (start) {
    led.toggle(Math.random(5), Math.random(5))
  }
})
\end{lstlisting}
\caption{\label{fig:example}Running example.}
\end{figure}

\subsection{Evaluation}

%we encourage the reader to choose a target
%from \url{www.makecode.com} and experiment with programming it, to appreciate the
%qualitative aspects of the platform, namely its simplicity and ease of use.
In this paper, we evaluate quantitative aspects of the platform
with respect to the devices from Table~\ref{table:devices}. In particular, we
consider:
\begin{itemize}
\item the time to compile Static TypeScript user code (to machine code) with respect
      to the GCC C/C++ toolchain, as well as the size of the resulting executable;
\item the time to load code onto a microcontroller using UF2, compared to standard bootloaders
      such as Arduino and ARM's DAPlink;
\item the performance of a set of small benchmarks, written in both Static TypeScript and C++,
      compiled with the \MC and GCC toolchains;
\item \emph{energy consumption: \CO vs. Arduino}
\item \emph{native code vs. bytecode} we
      evaluate memory consumption and code performance for native code generation
      vs. bytecode generation and interpretation on the Uno.
\end{itemize}

\flameon{TODO: we should present some of high-level experimental results here.}

All of the platform's components are open source on GitHub.
  
Sections~\ref{sec:makecode} to~\ref{sec:uf2} presents the four major components of the platform, top-down,
as referenced before. Section~\ref{sec:evaluate} evaluates the performance of the platform,
Section~\ref{sec:related} discusses related work, and Section~\ref{sec:conclude}
concludes.


\section{Example}

All exported functions with a block attribute will be available in the Block Editor.

\begin{lstlisting}
//% block
export function showNumber(v: number, interval: number = 150): void
{ }
\end{lstlisting}
If you need more control over the appearance of the block, you can specify the 
blockId and block parameters:
\begin{lstlisting}
//% blockId=device_show_number
//% block="show|number %v"
export function showNumber(v: number, interval: number = 150): void
{ }
\end{lstlisting}

% blockId is a constant, unique id for the block. This id is serialized in block code so changing it will break your users.
% block contains the syntax to build the block structure (more below).
% Other optional attributes can also be used:

% blockExternalInputs= forces External Inputs rendering
% advanced=true causes this block to be placed under the parent category’s “More…” subcategory. Useful for hiding advanced or rarely-used blocks by default
% Block syntax

\section{Blockly Overview}
\label{sec:blockly}

Not suprisingly, the core abstraction of Blockly is the \emph{Block},
which is used to represent statements, expressions, values and variables.  
Blocks have \emph{connectors} that allow them to be sequenced
horizontally or vertically in space.  In the default Blockly
layout, vertically sequenced blocks
represent program statements, while horizontally
sequenced blocks represent program expressions/values.
Blockly also supports the nesting of blocks, and so a collection of
blocks naturally represents an abstract syntax tree. 

While Blockly does attach a loose meaning to blocks,
their actual semantics given to blocks is entirely up to the Blockly developer.
We use the TypeScript language (and its type system) to 
give a more precise meaning to blocks. 

Block connectors are categorized and constrained as follows:
\begin{itemize}

\item \emph{previous and next connectors}, both optional, allow a block to be vertically sequenced - Blockly
  enforces that a block with a previous connector cannnot have an output connector; according
  to Blockly documentation, ``a statement block will usually have both a previous connection and 
  a next connection'';

\item a block may have a single \emph{output connector}, 
      which appears as a male jigsaw connector on a block's left-side; according to 
      Blockly documentation, ``Blocks with an output are usually called value blocks'';

\item a block may have multiple \emph{inputs}, which can appears as holes in a block
      or female jigsaw connectors on a block's right-side, about which we'll say more below. 
\end{itemize}

Block inputs come in three basic forms:
\begin{enumerate}
  \item \emph{fields}, which represent terminals (constants, literals, variables);
  \item \emph{value inputs} (receives value from output block) - value inputs
      and value blocks allow one to create expression trees;
  \item \emph{statement inputs}, allow one to create statement trees;
\end{enumerate}


\section{Primitive Values}

% The following types are supported in function signatures that are meant to be exported:

% boolean
% number (TypeScript)
% string (TypeScript) 

\section{API Function Calls}

API represented by
\begin{itemize}
  \item namespaces to organize set of related functions, 
       where namespaces map to Toolbox categories in Blockly;
       In particular, each top-level namespace is used to populate a category 
       in the toolbox. The name of the namespaces is capitalized for the toolbox. 

  \item void return type for a function means a call to the function is allowable in
      a statement context; non-void means expression context for call
  \item functions return a single value;
  \item parameters: optional parameters with default value; rest; 
\end{itemize}

% Block syntax

% The block attribute specifies how the parameters of the function will be organized to create the block.

% block = field, { '|' field }
% field := string
%     | string `%` parameter [ `=` type ]
% parameter = string
% type = string
% each field is mapped to a field in the block editor
% the function parameter are mapped in order to %parameter argument. 

% The loader automatically builds a mapping between the block field names and the function names.
% the block will automatically switch to external inputs when enough parameters are detected
% A block type =type can be specified optionally for each parameter. It will be used to populate the shadow type.
% Supported types

\section{Enumerations}

\section{Objects and Object Destructuring}
API represented by
\begin{itemize}
  \item set of classes, with
  \item constructors
  \item methods/fields
\end{itemize}

\section{Event Handlers and Callbacks}

% \item event-handlers via a callback function (callbacks only have parameters with primitive values);

A function $f$ that has an argument $g$ of function type (in last position) will have
that function argument (callback) converted into a statement input of $B(f)$.

If the callback $g$ has parameters, then
the best way to map that pattern to the blocks is by modifying
$g$ to have a single parameter with a class type that has the
various parameters as fields.
For example:

% export class ArgumentClass {
%     argumentA: number;
%     argumentB: string;
% }

% //% mutate=objectdestructuring
% //% mutateText="My Arguments"
% //% mutateDefaults="argumentA;argumentA,argumentB"
% // ...
% export function addSomeEventHandler((a: ArgumentClass) => void) { };

% In the above example, setting mutate=objectdestructuring will cause this API 
% to use Blockly “mutators” to let users change what parameters appear in the blocks. 
% Each parameter will be given an optional variable field in the block that defines a 
% variable that can be used within the callback. The variable fields compile to object
% destructuring in the TypeScript code. For example:

% addSomeEventHandler(({argumentA, argumentB}) => {

% })

% For an example of this pattern in action, see the radio.onDataPacketReceived block in the microbit target.

% In some cases it can be useful to change the runtime behavior of the API based on the properties 
% selected by the user. To enable that behavior, create an enum with entries that have the same names 
% as the argument object’s properties and add an extra parameter taking in an enum array to the API. 
% For example:

% export class ArgumentClass {
%     argumentA: number;
%     argumentB: string;
% }

% enum ArgNames {
%     argumentA,
%     argumentB
% }

% //% mutate=objectdestructuring
% //% mutateText="My Arguments"
% //% mutateDefaults="argumentA;argumentA,argumentB"
% //% mutatePropertyEnum="argNames"
% // ...
% export function addSomeEventHandler(args: ArgNames[], (a: ArgumentClass) => void) { };
% Note the mutatePropertyEnum attribute added to the comment annotations. The block for this API will look the same as the previous example but the compiled code will also include the arguments passed:

% addSomeEventHandler([ArgNames.argumentA, ArgNames.argumentB], ({argumentA, argumentB}) => {

% })
% The other attributes related to object destructuring mutators include:

% mutateText - defines the text that appears in the top block of the Blockly mutator dialog (the dialog that appears when you click the blue gear)
% mutateDefaults - defines the versions of this block that should appear in the toolbox. Block definitions are separated by semicolons and property names should be separated by commas




\section{Collections}

\section{Monads}

\section{Lexical Scoping}

\section{Attributes}

%% Bibliography
%\bibliography{paper}

%% Appendix
%\appendix
%\appendix
\pagebreak
\section{Static TypeScript Subtype Relation}


In STS, $S$ is a subtype of a type $T$ if one of the following is true:
\begin{itemize}
\item $S$ and $T$ are identical types;
\item $S$ is the Undefined type;
\item $S$ is the Null type and $T$ is not the Undefined type;
\item $S$ is an enum type and $T$ is the primitive type Number;
\item $S$ is a string literal type and $T$ is the primitive type String;
\item $S$ and $T$ are class types and all the following are true:
\begin{itemize}
  \item $S$ is derived from $T$ (via \emph{extends} clauses);
  \item checkProps($S$,$T$) holds;
\end{itemize}
\item $S$ is a class/record type and $T$ is a record type and
\begin{itemize}
  \item checkProps($S$,$T$) holds;
\end{itemize}
\item $S$ and $T$ are function types such that all the following hold:
\begin{itemize}
  \item $S$ has at least as many parameters as $T$;
  \item each parameter type in $T$ is a subtype of the corresponding parameter type in $S$;
  \item the result type of $T$ is Void, or the result type of $S$ is a subtype of that of $T$;
\end{itemize}
\item $S$ and $T$ are array types and all the following hold:
\begin{itemize}
\item $T$ has a numeric index signature with element type $U$,
    and $S$ has a numeric index signature with element type $V$
    such that $V$ is a subtype of $U$;
\item checkProps($S$,$T$) holds.
\end{itemize}
\end{itemize}

Given types $S$ and $T$, checkProps($S$,$T$) holds if for each property $N$ in $T$,
$S$ has a property $M$ where all of the following are true:
\begin{itemize}
\item $M$ and $N$ have the same name;
\item the type of $M$ is a subtype of the type of $N$;
\item $M$ and $N$ are both public, or $M$ and $N$ are both
      private (protected) and originate in the same declaration,
      or $N$ is protected and $S$ is a class derived from class $T$
\end{itemize}

\section{Artifact appendix}

Submission and reviewing guidelines and methodology: \\
{\em http://cTuning.org/ae/submission.html}

%%%%%%%%%%%%%%%%%%%%%%%%%%%%%%%%%%%%%%%%%%%%%%%%%%%%%%%%%%%%%%%%%%%%%
\subsection{Abstract}

This artifact allows others to reproduce the results seen in this paper for MakeCode and Codal, using the BBC micro:bit. The artifact contains an offline build environment for codal and MakeCode, allowing evaluators to test and build programs locally. In addition, we also provide espruino and micropython virtual machines to further increase repeatability of our results. Evaluators should download the virtual machine containing all pre-requisite tools, and use an oscilloscope to observe wave forms (used for timing) generated by the micro:bit, and a serial terminal to observe results reported from the micro:bit over serial.


\subsection{Artifact check-list (meta-information)}

{\small
\begin{itemize}
  \item {\bf Program:} MakeCode \& Codal
  \item {\bf Compilation:} arm-none-eabi-gcc
  \item {\bf Binary:} espruino, and micropython binaries included; others compiled during testing
  \item {\bf Run-time environment:} Codal
  \item {\bf Hardware:} BBC micro:bit
  \item {\bf Output:} Waveforms, and serial output
  \item {\bf Publicly available?:} Yes
  \item {\bf Artifacts publicly available?:} Yes
  \item {\bf Artifacts functional?:} Yes
  \item {\bf Artifacts reusable?:} Yes
  \item {\bf Results validated?:} Yes
\end{itemize}

%%%%%%%%%%%%%%%%%%%%%%%%%%%%%%%%%%%%%%%%%%%%%%%%%%%%%%%%%%%%%%%%%%%%%
\subsection{Description}

\subsubsection{How delivered}

The artifact is available on GitHub:\\[5pt]\url{https://lancaster-university.github.io/lctes-artefact-evaluation/}\\[5pt] And a virtual machine, based on debian, containing all the required software to reproduce our results is available here:\\[5pt]\url{https://drive.google.com/open?id=1nxiorz6NRqjen89G59RCOEMklqAyaUv7}

\subsubsection{Hardware dependencies}

\begin{itemize}
    \item A BBC micro:bit
    \item An oscilloscope
    \item A computer capable of running a virtual machine
\end{itemize}

\subsubsection{Software dependencies}

\begin{itemize}
    \item A virtual machine obtained from the URL above.
    \item A serial terminal.

\end{itemize}

%%%%%%%%%%%%%%%%%%%%%%%%%%%%%%%%%%%%%%%%%%%%%%%%%%%%%%%%%%%%%%%%%%%%%
\subsection{Installation}

Use virtual box to install the image located at:\\[5pt]\url{https://drive.google.com/open?id=1nxiorz6NRqjen89G59RCOEMklqAyaUv7}\\[5pt]
and the VirtualBox extension pack:\\[5pt]\url{https://www.virtualbox.org/wiki/Downloads}
%%%%%%%%%%%%%%%%%%%%%%%%%%%%%%%%%%%%%%%%%%%%%%%%%%%%%%%%%%%%%%%%%%%%%
\subsection{Experiment workflow}

Tests generally follow the following sequence of steps:

\begin{enumerate}
    \item Perform small program modifications.
    \item Compile the program.
    \item Transfer program to the micro:bit (flashing).
    \item Observe either a waveform generated by the micro:bit using an oscilloscope, or serial output from the micro:bit using a serial program.
\end{enumerate}

%%%%%%%%%%%%%%%%%%%%%%%%%%%%%%%%%%%%%%%%%%%%%%%%%%%%%%%%%%%%%%%%%%%%%
\subsection{Evaluation and expected result}

We expect the results to be the same as those reported in the paper. The observed waveforms may differ in time due to different compilers, oscilloscopes, and oscilloscope calibration.

%%%%%%%%%%%%%%%%%%%%%%%%%%%%%%%%%%%%%%%%%%%%%%%%%%%%%%%%%%%%%%%%%%%%%
\subsection{Experiment customization}

All tests provided have a clear set of corresponding instructions that evaluators should follow to observe the same results. Any steps involving customisation have been minimised.

%%%%%%%%%%%%%%%%%%%%%%%%%%%%%%%%%%%%%%%%%%%%%%%%%%%%%%%%%%%%%%%%%%%%%
\subsection{Notes}

The virtual machine contains a folder named `evaluators' which is placed in the home directory of the lctes user. The username for the virtual machine is: \textit{lctes} and the password is: \textit{lctes2018}. To become super user, type \textit{su} in a terminal, and enter the same password (\textit{lctes2018}).

Once logged in, and in the `evaluators' directory, you can view the tests as markdown files in the `docs' directory. Alternately, these markdown documents can also be viewed on the web by running `mkdocs serve' in the evaluators folder, or browsing to:\\[5pt]\url{https://lancaster-university.github.io/lctes-artefact-evaluation/}\\[5pt] Which is a pre-built, and hosted version produced from the same source.

We recommend that you add the micro:bit usb device using the machine settings tab in virtual box as shown in the image below:\\

\includegraphics[width=\columnwidth]{images/virtualbox.png}

We also have a convenience script for mounting a shared folder between the host and the vm. Simply create a shared folder named `lctes-vm-dir' and run `sh mount.sh' (contained in evaluators) as a super user to mount the shared folder to vb-share (also contained in evaluators). Shared folder creation in VirtualBox is pictured below:\\

\includegraphics[width=\columnwidth]{images/shared-folder.png}
%\section{USB Flashing Format}
\label{sec:uf2}

There are several ways of installing a user program in the non-volatile flash memory
of a microcontroller. In professional scenarios one typically uses a debugger
interface of the target, with a specialized debugger chip and a native application
talking to it. Hobbyist typically employ a custom protocol over RS-232 serial
line. This typically requires operating system drivers and a native application.

ARM mbed uses DAPLink firmware, which presents itself to an external computer
as a USB pen drive. No special drivers need to be installed, as operating
systems support pen drives out of the box. The USB pen drive protocol (USB Mass
Storage Class or MSC) is a block-level protocol (generally 512 bytes) with no
concept of files. 

DAPLink exposes a small virtual FAT file system, which
never changes due to OS writes - it contains an informational HTML file, but
the file allocation table and the root directory are otherwise empty.
When the OS tries to read a block, DAPLink computes what should be there, 
based on compiled-in contents of the HTML file.
On file system writes, DAPLink detects the Intel HEX format~\cite{IntelHEX}, 
decodes it and flashes the file's contents into the target microcontroller's memory. 
Other writes are ignored.

DAPLink needs to implement heuristics to deal with quirks of FAT file
system implementations in various operating systems (order of writes, various meta-data files
that are created and need to be ignored, etc.). It is acceptable since DAPLink runs
on a separate debugger chip, however in a single-chip setup it is quite heavy for a bootloader occupying 
flash space of a target MCU.

Instead of dealing with these quirks we decided to keep the virtual FAT architecture, but change 
the file format, so that MSC writes can be easily acted upon.
Flashing Format (\UF) files consist of one or more 512-byte self-contained blocks.
The blocks have magic numbers, the payload data to be written to flash,
and the address where it should be written.
Thus, on every 512-byte write via the USB MSC, the bootloader can quickly and easily
check if the block being written is part of a \UF file (by comparing magic numbers)
and if so, write it immediately. In fact, this can be implemented in a streaming
fashion with less than 100 bytes of RAM, as on the ATmega16u2 interface chip
on Arduino UNO.
The minimal implementation of the \UF bootloader is 1-2k of code, depending
on the MCU instruction set and USB interfaces.
The bootloader on SAMD21 (which is 8k of code) implements \UF flashing with read-back capability (ie.,
a \UF file is exposed in the virtual FAT that contains current content of the 
MCU flash; as \MC embeds source code in binaries this lets the user drag the current \UF file from
a board into \MC browser window to load the project), legacy serial bootloader for Arduino, and a custom USB HID protocol
for flashing with a native application (but no drivers).



\end{document}
