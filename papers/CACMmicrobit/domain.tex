\section{Physical Computing}
\label{sec:domain}

[In this section we show off the kind of projects made with the micro:bit]



% from Steve and Sue's white paper:

% We summarize the benefits of physical computing in the classroom thus:
%
% Motivation:
%
% Increased motivation for students, including those from diverse backgrounds, 
% because the learning experience and the outcome are visible not virtual. 
% This is especially true when a programming task delivers a practical, meaningful device.~\cite{XYZ}
%
% Tangibility & Interactivity:	
%
% The tangible nature of physical devices helps students make natural connections. 
% Iteratively debugging and refining tangible systems helps them better understand
% programming concepts and the software development process.~\cite{XYZ}
%
% Creativity:
%
% Students naturally relate to the physical nature of the task, unleashing creativity 
% in terms of what they build and thereby strengthening engagement with the task.~\cite{XYZ}
%
% Learning by doing:
%
% Physical computing projects promote trial-and-error because there are many ways to 
% achieve most goals rather than a single correct solution. This supports learning by 
% doing in an iterative fashion. 
%
% Collaboration:
% 
% Working with devices lends itself to group work – different roles include enclosure
% design, hardware interfacing, algorithm design and user interaction. Groups of students
% can readily cooperate (or compete!) because of the physical nature of challenges and tasks.
%
% Holistic View of Computing Education:
%
% Computer systems are comprised of hardware as well as software, and computer science is not
% just about programming. It is important for students to learn about the physical hardware 
% components of computer systems and how they work, especially given the emergence of the 
% internet of things (IoT).
%
% Engages the Whole Learner:
%
% The physical nature of the work engages the whole student – both their mind and their body,
% making the learning process a deep, immersive experience.

% Finally, it is worth noting that the benefits of physical computing aren’t limited to CS education. 
% There are diverse connections to other STEM subjects, such as the simulation of behaviour in biology, 
% the collection and analysis of measurements in physics and logical mathematical operations [7]. 
% Physical computing also connects into the arts and humanities, with application to topics ranging 
% from interactive art pieces to geography and dance [5], [7]. 
