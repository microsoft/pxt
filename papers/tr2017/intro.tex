\section{Introduction}
\label{sec:intro}

Over the last decade, microcontrollers, the workhorses of embedded systems, have become
central to efforts in making and education. For example, the Arduino project
(\url{www.arduino.cc})~\cite{buildingArduino2014},
started in 2003, created the Uno board using an 8-bit Atmel
AVR microcontroller. The Uno makes its microcontroller's I/O pins available via headers;
external hardware modules (shields) may be connected to these headers to extend
the Uno's capability. The Arduino ecosystem has grown tremendously in the past 15 years,
with the support of companies such as Adafruit Industries (\url{www.adafruit.com}) and
Sparkfun Electronics (\url{www.sparkfun.com}).

What has not changed much is the way microcontrollers are programmed,
which is with the C and C++ programming languages, as well as some assembly.
This is not a huge surprise, given the low-level nature of microcontroller programming,
where direct access to the hardware is needed. There generally is no operating
system running on such boards, as they have very little RAM (2Kb on the Uno, for example) and
lack memory protection hardware. What is more surprising about the Arduino platform is that
it:
\begin{itemize}
\item encourages the programmer to poll sensors,
leading to monolithic programs;
\item lacks any code ``intellisense'' or common interactive features of modern IDEs;
\item loads code onto the microcontroller using 1980's era bootloader technology.
\end{itemize}
On the other hand, on the web we find many excellent environments for introductory programming.
Visual block editors such as Scratch (\url{https://scratch.mit.edu/})~\cite{ScratchCACM2009,BlocksBeyondCACM2017}
and Blockly (\url{https://developers.google.com/blockly/})~\cite{Blocky2015}
allow the creation of programs without the possibility of syntax errors.
The programming models associated with Scratch and Blockly generally are
event based, freeing the programmer from the need to poll.
HTML, CSS and JavaScript allow a complete programming experience to be delivered as an interactive
web app, including editing with intellisense, code execution and debugging~\cite{Monaco}.

We have created a new programming platform that bridges the worlds of the microcontroller
and the web app. The major goals of the platform are to: (1)
make it simple to program microcontrollers using nothing more than a web app;
(2) allow a user's compiled program to be easily installed on a microcontroller;
(3) enable the safe addition of new of software/hardware components to a microcontroller.

\begin{figure*}[t]
  \includegraphics[width=5.5in]{reposFig.pdf}
  \caption{\label{fig:repos}Relationships between platform components/repos. Yellow boxes represent the \MC (PXT) components; blue
  boxes represent the \CO components; green boxes represent external components.}
\end{figure*}

The platform consists of a stack of four novel technologies, the subject of
this paper:
\begin{itemize}
\item \emph{\MC (\href{https://makecode.com}{makecode.com})}, a web app that supports both visual block programming and text programming,
via \emph{Static TypeScript}, with conversion between the two program representations

\item \emph{Static TypeScript}, a statically-typed subset of TypeScript (\url{www.typescriptlang.org}),
a gradually-typed superset of JavaScript, for fast execution on low-memory devices, with
a simple model for linking against pre-compiled C++;

\item \emph{\CO (the Component-oriented Device Abstraction Layer)}, an event driven, multi-threaded, C++ runtime environment that bridges the semantic gap between higher-level languages and the hardware,
modelling each hardware component as a software component.

\item \emph{USB Flashing Format} (UF2), a new file format designed for flashing microcontrollers over the Mass Storage
Class protocol; the format greatly speeds the installation of user
programs and is robust to differences in operating systems.
\end{itemize}
The \MC web app is the entry point of the platform, and has in-browser execution via a device simulator, as well as compilation to machine code and linking against a
pre-compiled C++ runtime (\emph{\CON}). No C/C++ compiler is invoked to compile user code and the result of compilation is a binary file that is ``downloaded'' from the web app to the user's
computer and then flashed to the microcontroller, with the aid of the \emph{UF2} file format.

These four advances enable beginners to get started programming microcontrollers from any modern web browser, and enable
hardware vendors to innovate and safely add new components to the mix using Static TypeScript, leveraging its
foreign function interface to C++.
Once the web app has been loaded, all the above functionality works offline (i.e., if the host machine loses its connection
to the internet).

Platform targets can be seen at \url{www.makecode.com}, where the \MC web app for a variety of boards is available,
including:
\begin{itemize}
\item the \emph{\href{https://microbit.org}{micro:bit}}, a Nordic nRF51822 microcontroller with Cortex-M0 processor, 256KB flash and 16KB RAM~\cite{microbitICSE2016};
\item Adafruit's \emph{\href{https:/adafruit.com/products/3333}{Circuit Playground Express}}, an Atmel SAMD21 microcontroller with Cortex-M0 processor, 256KB flash and 32KB RAM;
\item the \emph{\href{https://store.arduino.cc/usa/arduino-uno-rev3}{Arduino Uno}}, an Atmel ATmega328 microcontroller with AVR processor, 32KB fLash and 2KB RAM.
\end{itemize}

We encourage the reader to choose a target
and experiment with programming it, to appreciate the
qualitative aspects of the platform, namely its simplicity and ease of use.
This paper evaluates quantitative aspects of the platform:
compilation speed, code size, and runtime performance.  In particular, we
consider:
\begin{itemize}
\item the time to compile Static TypeScript user code (to machine code) with respect
      to the GCC C/C++ toolchain, as well as the size of the resulting executable;
\item the time to load code onto a microcontroller using UF2, compared to standard bootloaders
      such as Arduino and ARM's DAPlink;
\item the performance of a set of small benchmarks, written in both Static TypeScript and C++,
      compiled with the \MC and GCC toolchains;
\item \emph{energy consumption: \CO vs. Arduino}
\item \emph{native code vs. bytecode} we
      evaluate memory consumption and code performance for native code generation
      vs. bytecode generation and interpretation on the Uno.
\end{itemize}


\begin{figure*}[t]
      \includegraphics[width=5in]{screenSnapFig.pdf}
  \caption{\label{fig:screenSnap}Screen snapshot of the \MC web app.}
\end{figure*}

All of the platform's components are open source on GitHub.
Figure~\ref{fig:repos} lists the major GitHub repos of the platform
and the dependences between them. Green boxes represent repos external to the platform
(note: not all repos are represented in the figure).

The \MC framework
is at \emph{\href{https://github.com/microsoft/pxt}{pxt}} (PXT is the previous codename of \MCN).
A \emph{pxt-target} extends the framework to create an environment for a specific board. Targets
for the three previously mentioned boards are at:
~\emph{\href{https://github.com/microsoft/pxt-microbit}{pxt-microbit}},
~\emph{\href{https://github.com/microsoft/pxt-adafruit}{pxt-adafruit}}, and
~\emph{\href{https://github.com/microsoft/pxt-arduino-uno}{pxt-arduino-uno}}.
The latter two targets make use of a common set of libraries,
\emph{\href{https://github.com/microsoft/pxt-common-packages}{pxt-common-packages}},
which build upon \CON's microcontroller independent core abstractions at
~\emph{\href{https://github.com/lancaster-university/\CO-core}{\COLN-core}}.

Platform- and microcontroller-dependent specialization and optimizations for
the SAMD21 and Atmega328 microcontrollers can be found at
~\emph{\href{https://github.com/lancaster-university/codal-samd21}{\COLN-samd21}},
and
~\emph{\href{https://github.com/lancaster-university/codal-atmega328p}{\COLN-atmega328p}}.
Not shown in the figure, the SAMD21 repo uses another repo for
MBED-specific optimizations for the Cortex-M0 processor: \emph{\href{https://github.com/lancaster-university/codal-mbed}{\COLN-mbed}}.

The repo \emph{\href{https://github.com/lancaster-university/codal}{\COLN}} provides the
tools for compiling a \CO target.  As can be seen in Figure~\ref{fig:repos},
\CO abstracts over platform and microcontroller specific
implementations, while PXT abstracts over programming editors and languages.
The UF2 specification is at~\emph{\href{https://github.com/microsoft/uf2}{uf2}},
with implementations at~\emph{\href{https://github.com/microsoft/uf2-samd21}{uf2-samd21}}
and~\emph{\href{https://github.com/mmoskal/uf2-uno}{uf2-uno}}. These repos are not
shown in the Figure.
The pxt-microbit target uses a predecessor of the \CO called the DAL (at
~\emph{\href{https://github.com/lancaster-university/microbit-dal}{microbit-dal}}).

The rest of this paper presents the four major components of the platform,
top-down: Section~\ref{sec:makecode} presents the design
and implementation of the \MC framework, while
Sections~\ref{sec:sts},~\ref{sec:codal} and~\ref{sec:uf2} describe Static TypeScript,
the \CO C++ runtime, and the UF2 flashing format, respectively.
Section~\ref{sec:evaluate} evaluates the performance of the platform,
Section~\ref{sec:related} discusses related work, and Section~\ref{sec:conclude}
concludes.
