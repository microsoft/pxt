\section{Introduction}
\label{sec:intro}

Over the last decade, microcontrollers, the workhorses of embedded systems, have become 
central to efforts in making and education. For example, the Arduino project (\url{www.arduino.cc}), 
started in 2003, created the Uno using an 8-bit Atmel 
AVR microcontroller. The Uno makes most of its microcontrollers I/O pins available via headers;
external hardware modules (shields) may be connected to the Uno's headers to extend 
its capability.    The Arduino ecosystem has grown tremendously in the past 15 years, 
with the support of companies such as Adafruit Industries (\url{www.adafruit.com}) and 
Sparkfun Electronics (\url{www.sparkfun.com}).

What has not changed much is the way microcontrollers are programmed,
which is with the C and C++ programming languages, as well as assembly.   
This is not a huge surprise, given the low-level nature of microcontroller programming, 
where direct access to the hardware is needed. There generally is no operating 
system running on such boards, as they have very little RAM (2K for the Uno, for example) and 
lack memory protection hardware. What is more surprising about the Arduino platform is that:
\begin{itemize}
\item it encourages the programmer to use polling to interact with sensors, 
which leads to monolithic sequential programs;
\item its IDE lacks any code ``intellisense'' or common interactive features of modern IDEs;
\item it loads code onto the microcontroller using 1980s era bootloader technology.
\end{itemize}
As a result, it is not simple to get started with Arduino systems, of which there are many. 
On the other hand, on the web we find many excellent environments for introductory programming. 
Visual block editors such as Scratch (\url{https://scratch.mit.edu/})~\cite{ScratchCACM2009,BlocksBeyondCACM2017} 
and Blockly (\url{https://developers.google.com/blockly/})~\cite{Blocky2015}
allow the creation of programs without the possibility of syntax errors. 
HTML, CSS and JavaScript allow a complete programming experience to be delivered as an interactive 
web app, including editing with intellisense, code execution and debugging. (While the Arduino IDE recently 
has been ported to the web, it lacks many of the above features and requires a web connection to a server which runs 
a C/C++ compile tool chain to compile user code.) The programming models associated with these environments are 
event based, freeing the programmer from the need to poll.

We have created a new programming platform that bridges the worlds of the microcontroller
and the web app. The major goals of the platform are to: (1)
make it simple to program microcontrollers using nothing more than a web app;
(2) allow a user's compiled program to be easily installed on a microcontroller;
(3) enable the safe addition of new of software/hardware components to a microcontroller.

\begin{figure*}[t]
  \includegraphics[width=5.5in]{reposFig.pdf}
  \caption{\label{fig:repos}Relationships between platform components/repos. Yellow boxes represent the \MC (PXT) components; blue
  boxes represent the CODAL components; green boxes represent external components.}
\end{figure*}

The platform builds on stack of four novel technologies that are the subject of
this paper. The entry point of the platform is \emph{\MC} (\href{https://makecode.com}{makecode.com}),
a web app that supports both visual block programming and text programming,
via \emph{Static TypeScript}, with conversion 
between the two program representations. The web app has in-browser execution 
via a device simulator, as well as compilation to machine code and linking against a 
pre-compiled C++ runtime (\emph{CODAL}). No C/C++ compiler is invoked to compile user code. 
The result of compilation is a binary file that is ``downloaded'' from the web app to the user's 
computer and then flashed to the microcontroller, with the aid of the \emph{UF2} file format and
supporting firmware. The other three major technologies in the stack are:
\begin{itemize}

\item \emph{Static TypeScript} is a statically-typed subset of TypeScript (\url{www.typescriptlang.org}), 
a gradually-typed superset of JavaScript, for fast execution on low-memory devices 
and a simple model for linking against pre-compiled C++; 

\item \emph{CODAL}, the Component-oriented Device Abstraction Layer, is a new C++ library that maps 
each hardware component of a device to one or more software components that communicate over a message bus and
schedule event handlers to run non-preemptively on fibers. 

\item \emph{USB Flashing Format} (UF2) is a new file format designed for flashing microcontrollers over the Mass Storage
Class (removable USB pen drive) protocol; the format and its supporting firmware greatly speeds the installation of user 
programs and is robust to difference in operating systems. 
\end{itemize}
These advances enable beginners to get started programming microcontrollers from any modern web browser, and enable
hardware vendors to innovate and safely add new components to the mix using Static TypeScript, leveraging its
foreign function interface to C++. Once the web app has been loaded, 
all the above functionality works offline (i.e., if the host machine loses its connection 
to the internet).
All of the above components are open source under the MIT/Apache licenses, as detailed below. 

Platform targets can be seen at \url{www.makecode.com}, where the \MC web app for a variety of boards is available, 
including:
\begin{itemize}
\item the \emph{\href{https://microbit.org}{micro:bit}}, a Nordic nRF51822 microcontroller with Cortex-M0 processor, 16K RAM~\cite{microbitICSE2016};
\item Adafruit's \emph{\href{https:/adafruit.com/products/3333}{Circuit Playground Express}}, an Atmel SAMD21 microcontroller with Cortex-M0 processor, 32K RAM;
\item the \emph{\href{https://store.arduino.cc/usa/arduino-uno-rev3}{Arduino Uno}}, an Atmel ATmega328 microcontroller with AVR processor, 2K RAM.
\end{itemize}

We encourage the reader to choose a board and experiment with programming it, to appreciate the 
qualitative aspects of the platform: its simplicity and ease of use.  In this 
paper, we evaluate quantitative aspects of the platform: 
compilation speed, code size, and runtime performance.  In particular, we evaluate:
\begin{itemize}
\item the compile time of Static TypeScript compile/link of user code (to machine code) with respect 
      to the GCC C/C++ toolchain, as well as the size of the resulting executable;
\item the time to load code onto a microcontroller using UF2, compared to standard bootloaders; 
\item The performance of a set of small benchmarks, written in both Static TypeScript and C++,
      compiled with the \MC and GCC toolchains, as well as the performance of device drivers
      written in Static TypeScript compared to their C++ counterparts;
\item \emph{energy consumption: CODAL vs. Arduino}
[evaluate with respect to the popular Arduino toolset, for boards with 8-bit (AVR) and 32-bit (Cortex-M0) microcontrollers. 
Summary of evaluation]
\end{itemize}

All of the platform's components are open source on GitHub, 
which also is true of external components that the platform
builds on (TypeScript, Blockly, Monaco). Figure~\ref{fig:repos}
lists the GitHub repos and the dependences between them. Green
boxes represent repos external to the platform.

\begin{figure*}[t]
      \includegraphics[width=5in]{screenSnapFig.pdf}
  \caption{\label{fig:screenSnap}Screen snapshot of the \MC web app.}
\end{figure*}

The \MC framework 
is at \emph{\href{https://github.com/microsoft/pxt}{pxt}} (PXT is the previous codename of \MC). 
A PXT \emph{target} extends the framework to create an environment for a specific board. Targets
for the three previously mentioned boards are at: 
~\emph{\href{https://github.com/microsoft/pxt-microbit}{pxt-microbit}}, 
~\emph{\href{https://github.com/microsoft/pxt-adafruit}{pxt-adafruit}}, and
~\emph{\href{https://github.com/microsoft/pxt-arduino-uno}{pxt-arduino-uno}}.
The latter two targets make use of a common set of libraries (packages) at
\emph{\href{https://github.com/microsoft/pxt-common-packages}{pxt-common-packages}},
which build upon CODAL's microcontroller independent core abstractions at
~\emph{\href{https://github.com/lancaster-university/codal-core}{codal-core}}.  
Platform- and microcontroller-dependent specialization and optimizations for ARM's Mbed platform, 
and SAMD21 and Atmega328 microcontrollers can be found at
~\emph{\href{https://github.com/lancaster-university/codal-mbed}{codal-mbed}},
~\emph{\href{https://github.com/lancaster-university/codal-samd21-mbed}{codal-samd21-mbed}}, and
~\emph{\href{https://github.com/lancaster-university/codal-atmega328p}{codal-atmega328p}}.
The repo \emph{\href{https://github.com/lancaster-university/codal}{codal}} provides the
build tooling for compiling a CODAL target.  As can be seen in Figure~\ref{fig:repos}, 
CODAL (the blue boxes) abstracts over platform and microcontroller specific
implementations (for example, MBED), while \MC/PXT abstracts over programming editors
and languages (Blockly, Monaco, and TypeScript).
The UF2 specification is at~\emph{\href{https://github.com/microsoft/uf2}{uf2}},
with implementations at~\emph{\href{https://github.com/microsoft/uf2-samd21}{uf2-samd21}}
and~\emph{\href{https://github.com/mmoskal/uf2-uno}{uf2-uno}}. These repos are not
shown in the Figure. 

The pxt-microbit target uses a predecessor of the CODAL called the DAL (at
~\emph{\href{https://github.com/lancaster-university/microbit-dal}{microbit-dal}}).

The rest of this paper is organized as follows. Section~\ref{sec:makecode} presents the design and implementation of the \MC framework. 
Sections~\ref{sec:sts},~\ref{sec:codal} and~\ref{sec:uf2} describe Static TypeScript, the CODAL C++ runtime, and the UF2 flashing format,
respectively.  Section~\ref{sec:evaluate} evaluates the performance of the platform and
Section~\ref{sec:related} discusses related work. Finally, Section~\ref{sec:conclude} concludes and speculates about future directions. 
