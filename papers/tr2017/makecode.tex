\section{\MC: Design and Implementation}
\label{sec:makecode}

Figure~\ref{fig:screenSnap} shows a screen snapshot of the \MC web app for the micro:bit
with the main parts labelled: 
(A) the menu bar allows switching between views of blocks and JavaScript;
(B) the simulator provides feedback on user code executed in the browser;
(C) the toolbar provides access to device-specific APIs and programming elements;
(D) the programming canvas; 
(E) the download button invokes the compiler to produce a binary executable.

The \MC web app encapsulates all the components needed to deliver a programming experience 
for microcontroller based devices, free of the need for a C++ compiler for the compilation of user 
code.
The web app is written in TypeScript and also incorporates the TypeScript compiler and 
language service as well. 
The app is built from a target (e.g., \emph{\href{https://github.com/microsoft/pxt-microbit}{pxt-microbit}})
which extends the core framework (\emph{\href{https://github.com/microsoft/pxt}{pxt}}).
The remaining subsections describe the parts of Figure~\ref{fig:makecode}, 
which shows the high-level components of the web app.

\begin{figure*}[t]
    \includegraphics[width=5.5in]{makecodeFig.pdf}
\caption{\label{fig:makecode}\MC Architecture}
\end{figure*}

\subsection{Device Runtime and Shim Generation}

An \MC target is defined, in part, by its device runtime, which can be a combination of C++ 
and Static TypeScript (STS) code. The C++ runtime for the target microcontroller is precompiled 
and stored in the cloud. The runtime binary also is stored in the HTML5 application cache (with 
other assets) so that the web app can function when the browser is offline. Additional runtime
components may be authored in STS, which allows the device runtime to be updated without the
use of C++, and permits components of the device runtime to be shared by both the device
and simulator runtimes. The ability to author the device runtime in both STS and C++ is
a unique aspect of \MC's design.

Whether runtime components are authored in C++ or STS, all runtime APIs are exposed as fully-typed
TypeScript definitions, as described later. A full-typed runtime improves the end-user experience 
by making it easier to discover APIs; it also enables the type inference provided by the TypeScript 
compiler to infer types for (unannotated) user JavaScript programs.

\MC supports a simple foreign function interface from TypeScript to C++ based on namespaces,
enumerations, functions, and basic type mappings. \MC uses top-level namespaces (in both C++ and
TypeScript) to organize sets of related functions.  Preceding a C++ namespace, enumeration, or function
with \emph{//\%} indicates that \MC should map the C++ construct to TypeScript.
Within the \emph{//\%} comment, attributes are used to define the visual appearance for that
language construct, such as for the LED namespace in the micro:bit target:

\begin{lstlisting}
//% color=#5C2D91 weight=97 icon="\uf205"
namespace led { 
...
\end{lstlisting}

Figure~\ref{fig:screenSnap}(C) shows the toolbox of API and language categories, where the LED
namespace can been seen. Here is the C++ file defining the micro:bit's led namespace and its functions:
~\href{https://github.com/Microsoft/pxt-microbit/blob/master/libs/core/led.cpp}{pxt-microbit/libs/core/led.cpp}.

Mapping of functions and enumerations between C++ and TypeScript is straightforward
and performed automatically by \MC. 
Here's an example of the C++ function plot in the led namespace that wraps a more
complex function call of the underlying device runtime to set/plot an LED in the micro:bit display:

\begin{lstlisting}
//% blockId=device_plot block="plot|x %x|y %y"
//% x.min=0 x.max=4 y.min=0 y.max=4
void plot(int x, int y) {
    uBit.display.image.setPixelValue(x, y, 1);
}
\end{lstlisting}

We'll describe the attribute definitions in the \emph{//\%} comment in the next section. 
\MC uses a TypeScript declaration file to describe the TypeScript elements corresponding
to C++ namespaces, enumerations and functions.  We call such files shim files.
Since the C++ plot function is preceded by a \emph{//\%} comment, 
\MC adds the following TypeScript declaration to the shim file (shims.d.ts) and copies
over the attribute definitions in the comment. \MC also adds an attribute definition mapping
the TypeScript shim to its C++ function (shim=led::plot):

\begin{lstlisting}
//% blockId=device_plot block="plot|x %x|y %y
//% x.min=0 x.max=4 y.min=0 y.max=4 shim=led::plot
function plot(x: number, y: number): void;
\end{lstlisting}

Here is the shim file generated from C++ micro:bit device runtime:
\href{https://github.com/Microsoft/pxt-microbit/blob/master/libs/core/shims.d.ts}{pxt-microbit/libs/core/shims.d.ts}.

To support the foreign function interface, \MC defines a mapping between C++ and TypeScript types.
Boolean and void have straightforward mappings from C++ to TypeScript (bool $\rightarrow$ boolean, void $\rightarrow$ void). 
As JavaScript only supports number, which is a C++ float/double, \MC uses TypeScript's support
for type aliases to name the various C++ integer types commonly used for microcontroller programming
(int32, uint32, int16, uint16, int8, uint8). 
% don't use int, unsigned etc. they are actually different sizes on different compilers for AVR
This is particularly useful for saving space on 8-bit architectures such as the AVR. 

\MC allows a set of C++ functions with the same first parameter (of type Foo) to be
exposed as a TypeScript interface Foo as follows: this set of C++ functions must be grouped
inside a namespace of the name FooMethods.  See, for example, how a C++ buffer abstraction is exposed:
\href{https://github.com/Microsoft/pxt-microbit/blob/master/libs/core/buffer.cpp}{pxt-microbit/libs/core/buffer.cpp}.
You can find the resulting TypeScript Buffer interface in the previously referenced shim file for the micro:bit.

\MC includes reference counted C++ types for strings, lambdas (with
up to three arguments and a return type) and collections.  
\MC does not yet include garbage collection, so advanced users who create cyclic
data structures must be careful to break cycles in order to ensure complete deallocation. 

\subsection{Block Metadata Generation}

Both C++ and TypeScript APIs can be specially annotated (minimally via 
\emph{//\% block}) so that the \MC compiler generates the needed
Blockly metadata to expose an API as a visual block. So, to expose the previously
encountered plot function as a visual block (as well as a TypeScript function), one simply needs:
\begin{lstlisting}
//% block
void plot(int x, int y) { . . . }
\end{lstlisting}

Additional attribute definitions can provide text descriptions for the block, project function
parameters (thus simplifying the API available via Blockly), and describe other visual/functional
characteristics of the block.  \MC uses the types of function parameters to select appropriate
Blockly widgets.  For example, an enumeration is represented by a dropdown menu in blocks.
For more information on the block-specific annotations, see 
\url{https://makecode.com/defining-blocks}. 
\MC's support for Blockly means that for the common case, the target developer doesn't need
to know anything about the Blockly framework.  For more sophisticated needs, one can directly access
the Blockly framework. 

\subsection{Editors and Code Conversion}

\MC uses the Blockly (\emph{\href{https://github.com/google/blockly}{blockly}}) and Monaco 
(\emph{\href{https://github.com/Microsoft/monaco-editor}{monaco-editor}}) editors to allow the user to code with
visual blocks or TypeScript. The editing experience is parameterized by the full-typed device
runtime, which provides a set of categorized APIs to the end-user, based on namespaces, as
previously described. These APIs are visible in both editors via a toolbox to the immediate
left of the programming area. The Blockly and Monaco toolboxes show the same set of APIs, to
help in transition from coding with blocks to coding with JavaScript. More advanced TypeScript
APIs can be discovered in Monaco via code intellisense.

The Blockly program representation is compiled to Static TypeScript in a syntax-directed manner
(see \emph{\url{https://github.com/Microsoft/pxt/tree/master/pxtblocks}{pxtblocks}}). A key issue is the need for
type inference on the Blockly representation, as variables generally are defined and used without
being declared in Blockly. \MC uses a simple unification type inference to assign a
unique type to each variable.  
% this may not be possible:
%In the future, we expect to use TypeScript's type inference instead
%and eliminate the need for separate type inference over the Blockly representation. 
TypeScript supports programming constructs that are not available in Blockly, such as classes.
Such constructs are converted into grey uneditable blocks in Blockly, with the construct's program
text intact. This means \MC always can decompile a TypeScript program to Blockly and then recover
the program text of the grey blocks when converting from Blockly back to TypeScript
 (see \emph{\href{https://github.com/Microsoft/pxt/blob/master/pxtcompiler/emitter/decompiler.ts}{decompiler.ts}}). 

\subsection{Compilation Pipeline}

\MC first invokes the TypeScript language service to perform type inference and type checking on the 
user's program, using the TypeScript declaration files for the device runtime.   It then checks that the
user's program is within the STS subset through additional syntactic and type checks over the adorned
abstract syntax tree (AST) produced by the language service (detailed in Section XYZ).  Assuming all the
above checks pass, \MC then performs tree shaking and compilation of the AST of the user code and
device runtime to an intermediate representation (IR) that makes explicit: labelled control flow among a
sequence of instructions with conditional and unconditional jumps; heap cells; field accesses; store operations,
and reference counting.

There are four backends for code generation from the IR. The first backend simply generates JavaScript,
for execution against the simulator runtime.  A second backend generates assembler, parameterized by a
processor description.  Currently supported processors include ARM's Cortex class (Thumb instructions)
and Atmel's Atmega class (AVR instructions). A separate assembler, also parameterized by an instruction
encoder/decoder, generates machine code and resolves runtime references, producing a binary executable.
A third backend generates bytecode instructions and a fourth backend targets C\#. 
The compiler chain
can be found at \emph{\href{https://github.com/Microsoft/pxt/tree/master/pxtcompiler/emitter}{pxtcompiler/emitter}} and 
\emph{\href{https://github.com/Microsoft/pxt/tree/master/pxtlib/emitter}{pxtlib/emitter}}.


\subsubsection{Asynchronous Functions}

An important part of the compilation process is to allow users to call asynchronous 
TypeScript functions (identified through the \emph{//\% async} annotation) 
as if they were blocking functions.  
For execution in the browser,
every function is compiled so that it can be suspended (at the return of a call) and later resumed (at the same point). 
The default behavior at a suspension point is to immediately resume execution.  For a call to an async function,
the default behavior is overridden by the compiler, which suspends execution of the current function. 
Upon completion of the async function call, the current function then is resumed. Such async functions are written
by runtime developers, not end-users, and greatly simplify the JavaScript
programming model for end-users. For example, although the JavaScript simulator runtime uses promises to 
achieve asynchronous execution in a single-threaded context, these promises are hidden from the end user. 
The CODAL device runtime supports fibers with the ability to pause, so for compilation to a device, 
the compiler simply emits a call to pause at a suspension point. 

\emph{TODO: need a small example to clarify how it all works}

% tball: I added TypeScript above to make it clear 
%\emph{TODO: The async annotation is only relevant for the JS code. In ARM/AVR it doesn't do anything.
%One way to say it is to say it's there to simulate cooperative multi-threading in JS, since
%it is already implemented in C++ by CODAL.}

\subsubsection{Untagged and Tagged Strategies}

The \MC compiler supports the Static TypeScript language subset described in Section~\ref{sec:sts},
with two compilation strategies: untagged and tagged. Under the untagged strategy,
a JavaScript number is interpreted as a C++ int by default and the type system is used
to statically distinguish primitive values from boxed values. As a result, the untagged
strategy is not fully faithful to JavaScript semantics: there is no support for floating
point and the Null and Undefined types are represented by the default integer value of zero.
This strategy is used for the micro:bit and Arduino Uno targets. 

In the tagged strategy, numbers are either tagged 31-bit signed integers, or if they do not fit, 
boxed doubles. Special constants like false, null and undefined are given special values 
and can be distinguished. The tagged execution strategy has the capability to fully support
JavaScript semantics. This strategy is used for all SAMD21 targets.

\subsection{Simulator}

A \MC target can provide an alternate TypeScript implementation for each API in the device runtime, for use in the device
simulator. As this code runs in the web browser (not on the actual device) and manipulates the DOM, the developer is free to
use all of TypeScript/JavaScript's features. (As an aside, \MC also support ``simulator-only'' targets that have no 
associated device; in these cases, the ``device runtime'' is defined solely by the simulator APIs.) 

The simulator allows the user to experience the basic functions of the device in the browser and to test their code
before deploying it to the actual device. The simulator has proxy widgets for sensors such as accelerometer (mouse motion),
temperature and light, allowing the user to control the sensor's value.  The simulator only provides basic functionality
and is far from a complete device emulation.   For example, it is not possible for the user to simultaneously modify two
inputs to the simulated device, while it is possible with the actual device (i.e., shaking it to change the accelerometer
reading while pushing one of the device's buttons).

%\MC provides various components to make device simulators easier to build: board, parts, wiring, etc.

\subsection{Packages and Custom Editors}

Packages are \MC's dynamic/static library mechanism for extending a target (by adding new code/data to the device
and simulator runtimes, as well as accompanying documentation). The following package extends the micro:bit target so
that the micro:bit can drive a NeoPixel strip of RGB LEDs: \url{https://github.com/Microsoft/pxt-neopixel}. 

To see how this
package is surfaced to the end-user, visit \url{http://makecode.microbit.org/} and select the ``Add Package'' option from the
gear menu; you will see the package ``neopixel'' listed in the available options. If you click on it, a new block category
named ``Neopixel'' will be added to the editor. In this scenario, PXT dynamically loads the (white listed) neopixel 
package directly from GitHub, compiles it and incorporates it into the web app. Packages also can be bundled with a web
app (the analog of static linking).  

For packages containing C++ files, a new C++ runtime has to be compiled in the cloud.
This is achieved by collecting all C++ files (in all packages plus the CODAL runtime),
computing a hash, checking in the local in-browser cache if such a runtime was retrieved
before, and otherwise asking the cloud service to compile it.
Of course, the cloud service will return a cached copy, if the same set of C++
files was compiled before. The cache hit rates, in both the browser cache
and the cloud cache, are very high. 
The first hit rate is high because the cloud needs to be consulted
only when a new package is added to the project, and this particular combination 
of packages was never used before on current machine.
The second hit rate is high because people rarely combine many packages (due to
hardware and memory constraints), and there are only so many of them.
