\section{Conclusion}
\label{sec:conclude}

\begin{itemize}
    \item Introducing TS and Blocks to this domain brings a range of immediate, 
        well understood benefits to users (Ease of programming, memory safety, 
        removal of syntactic errors, simplified event driven programming models). 
    \item Our combination of native C++ and TS also allows for low level optimisations 
          by users if they need it.
    \item Our software only approach means we have a wide platform base – 
         we are able to run on a diverse range of small, simple MCUs.
    \item Native compiled approach also promises performance advantages over 
          an interpreted bytecode VM approach.
\end{itemize}


% Hardware partners already have started to create \MC packages for the micro:bit.
% Seeed Studio (\url{https://www.seeedstudio.com/}) has created packages to add its Grove components to a micro:bit.
% Grove components are accessed via the I2C serial protocol, supported by the micro:bit device runtime.
% All micro:bit packages for the Grove components are authored in Static TypeScript (gesture, ultrasonic-ranger,
% 4-digital-display, two-led-matrix). These packages can be found under GitHub user ``Tinkertanker'', prefixed with
% ``pxt-''. Sparkfun has created \MC packages for its micro:bit shields (GitHub user ``sparkfun'').

% https://github.com/Tinkertanker/pxt-ssd1306-microbit
% https://github.com/Tinkertanker/pxt-ir-microbit 
% https://github.com/Tinkertanker/pxt-ky040-microbit
% https://github.com/Tinkertanker/pxt-ds1307-microbit
% Sparkfun
% •	https://github.com/sparkfun/pxt-weather-bit
% •	https://github.com/sparkfun/pxt-moto-bit 
% •	https://github.com/sparkfun/pxt-gamer-bit 
% Common packages
% •	https://github.com/Microsoft/pxt-common-packages
% •	Structure:
% o	Libs/package/
% [talk about C++] Various examples.
% [more about customer editor associated with a package]
