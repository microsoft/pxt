\section{Related Work}
\label{sec:related}

Related work breaks into three parts, following the
flow of the paper: the programming of microcontrollers,
type systems for JavaScript, and operating systems
for microcontrollers.

\subsection{Programming microcontrollers}

There are four basic ways to program microcontrollers,
(ordered roughly from lower to greater resource usage):
\begin{itemize}
\item[1.] a program executing on a host computer sends commands (via some communication protocol) 
to a fixed program loaded onto the microcontroller (e.g., Scratch's
device extensions~\cite{ScratchCACM2009}, Arduino's Firmata library~\cite{Firmata}).
\item[2.] a binary for the microcontroller is compiled on a host computer and loaded onto the microcontroller -
this is the common method for Arduino~\cite{buildingArduino2014};
\item[3.] bytecode is compiled on the host and loaded onto microcontroller, where it is interpreted
(e.g., Java for embedded systems~\cite{ClausenTOPLAS};
\item[4.] a complete compiler and virtual machine are put on the microcontroller, 
as with MicroPython~\cite{MicroPython} and Espruino~\cite{espruinoBook};
% 5. an embedded operating system
\end{itemize}

In the educational and maker settings, microcontrollers often are embedded in projects,
option 1 has limited utility, as in such projects the microcontroller is powered by battery and
is not connected to a host.

For these reasons, our platform supports (2) machine code and (3) byte code generation 
from JavaScript on the host, with reliance on a code loader (simplified
by use of \UF format and supporting firmware) to place the user's executable 
into flash memory.  The production of the entire executable file
takes places solely within the web app, which is a unique aspect of our platform.
The platform also supports partial flashing of just user code, 
when it can be determined that the proper device runtime already 
is resident in the microcontroller's flash.

As demonstrated in the evaluation, our approach to code generation and runtime
has significant performance advantages over MicroPython and Espruino.
This is due to a number of factors, the most significant of which is our use of
Static TypeScript. In practice, our hardware partners and users have had no
trouble with this limitation.  Another significant difference is our support
for non-preemptive concurrency via fibers (co-routines). 


% https://github.com/jerryscript-project

% JerryScript is a lightweight JavaScript engine for resource-constrained devices
% uch as microcontrollers. It can run on devices with less than 64 KB of RAM and
% less than 200 KB of flash memory.

% Key characteristics of JerryScript:

% Full ECMAScript 5.1 standard compliance
% 160K binary size when compiled for ARM Thumb-2



% https://github.com/cesanta/mongoose-os
% - Supported microcontrollers: CC3220, CC3200, ESP32, ESP8266, STM32F4;
% https://github.com/cesanta/mjs

% On 32-bit ARM mJS engine takes about 25k of flash memory, 
% and less than 1k of RAM (see intro article). mJS is part of MongooseOS,
% where it enables scripting for IoT devices.

% Restrictions
% No closures, only lexical scoping (i.e. nested functions are allowed).
% No new. In order to create an object with a custom prototype, use Object.create(), which is available.
% No getters, setters, classes, template strings.

\subsection{Types for JavaScript}

By design, TypeScript does not provide a type soundness guarantee~\cite{BiermanAT14}.
Type systems with soundness guarantees for JavaScript have as their
goals: (1) handling arbitrary JavaScript programs; 
(2) providing efficient execution for language subsets.

Safe TypeScript~\cite{SafeTypeScript15} separates the worlds of typed and untyped 
objects: the \emph{dynamic} type is an implicit union type of all \emph{known} static type, 
while Any is an \emph{unknown} type, coming from the untyped JavaScript world. Type tags 
are needed to distinguish the two worlds; Safe TypeScript also adds extra runtime checks
to prevent untyped JavaScript from destroying the type invariants of typed JavaScript.  

StrongScript~\cite{StrongScriptECOOP15} extends TypeScript with a type constructor (!)
for \emph{concrete} types, which allows the programmer to choose between untyped
code, optionally-typed code, and concretely typed code, which provides traditional
type soundess guarantees. As with TypeScript, every JavaScript program is a StrongScript program. 

Static TypeScript differs from these efforts by outlawing untyped or optionally typed
code: it is non-goal of Static TypeScript to support arbitrary JavaScript programs.
In this sense, Static TypeScript can be seen to be StrongScript where every variable and 
expression has a concrete (!) type. As in StrongScript, classes are nominally typed,
which permits a more efficient and traditional property lookup for class instances. 
Static TypeScript goes further by outlawing downcasts (necessary for our untagged 
implementation).

Static TypeScript could benefit by enlarging the set of types
that can be inferred, so that more JavaScript idioms could be allowed. 
Chandra's work on type inference for static compilation of 
JavaScript~\cite{ChandraOOPSLA2016} is one path forward. 

\subsection{Operating systems}

Typically written in C and/or C++, platforms for microcontroller programming all share 
a common design goal: to support developers by providing primitives and programming 
abstractions. Platforms can range from simple C++ classes that control hardware, 
to real-time operating systems (RTOSs) with scheduling and memory management.

Arduino~\cite{buildingArduino2014} is an example of a simple platform where the developer uses 
high-level APIs to control hardware; there is no scheduler, and memory management 
is discouraged through a heavy emphasis on global variables.  The Arduino programming 
model is based on polling: an Arduino ``sketch'' is a program template that consists 
of a setup procedure, for initialization of data structures, and a loop procedure; 
programmers implement the body of \textit{setup} and \textit{loop}, where they explicitly 
poll the state of the sensors or the microcontroller's pins.

At the other end of the spectrum is ARM's Mbed OS~\cite{ARMmbed}, which is aimed at developers who are familiar 
with non-preemptive scheduling, memory management, and condition synchronisation; a more 
complex environment. A number of high-level drivers are provided and the programming model 
is determined by the developer, commonly event or interrupt driven.

Although platforms like Arduino and Mbed OS are widely used by C/C++ developers, 
higher-level languages opt for virtual machines to execute on microcontrollers 
as neither extreme aligns with the programming models seen in higher level languages: 
Mbed OS is pre-emptive, JavaScript is not; Arduino is based on monolithic polling. 

The goals of \CO are to: (1) bridge the semantic gap between 
higher-level languages and the hardware in a memory, energy and instruction 
efficient way; and (2) support a number of platforms with various
resource contraints and capabilities.

Unlike Mbed, which is designed to make all devices look programmatically alike,
\CO enables devices with additional on-chip hardware to specialise device drivers 
with APIs that can then be used by the end application developer.

% https://state-machine.com/arduino/ 
% http://tinyos.stanford.edu/tinyos-wiki/index.php/TinyOS_Overview 

