\begin{abstract}

    Across the globe, it is now commonplace for educators to engage in the making (design and development) of embedded systems in the classroom to motivate and excite their students. This new domain brings its own set of unique requirements. Historically, embedded systems development requires: knowledge of low-level programming languages; local installation of compilation toolchains, device drivers, and applications on PCs; and constant tethered connections to PCs in order for programs to execute. For students and educators, these requirements introduce unsurmountable barriers and restrictions in the classroom.

    This paper presents a new programming platform that makes it easier for students and educators to create applications for embedded systems using high level languages. We leverage today's lightweight tooling to allow code to be developed without \emph{any installation}, requiring only a computer with a modern web browser and a USB port. Finally, we describe architectural decisions that we have taken to facilitate long, free running Internet of Things applications, enabling new classroom scenarios for students and educators to explore.

    With the predicted growth of the Internet of Things we believe that a system such as ours, that enables students and educators from diverse backgrounds to develop embedded applications quickly, will become increasingly valuable. After describing our platform in detail, we evaluate performance on a range of modern low-end embedded systems, from 8-bit to 32-bit cores.

\end{abstract}
