\begin{abstract}

    Across the globe, it is now commonplace for educators to engage in the making (design and development) of embedded systems in the classroom to motivate and excite their students. This new domain brings its own set of unique requirements. Historically, embedded systems development requires knowledge of low-level programming languages, local installation of compilation toolchains, device drivers and applications. For students and educators, these requirements can introduce unsurmountable barriers and restrictions.

    We present the motivation, requirements, implementation and evaluation of a new programming platform that enables novice users to create software for embedded systems. This platform consists of two major components: Microsoft MakeCode
    (\url{www.makecode.com}), a web app that encapsulates an entire beginner IDE for embedded
    systems; CODAL, an efficient component-oriented C++ runtime providing high-level programming abstractions. 
    We demonstrate how these two technologies together provide a highly accessible, cross-platform, installation-free programming experience for devices such as the BBC micro:bit and Adafruit Circuit
    Playground Express.
\end{abstract}
