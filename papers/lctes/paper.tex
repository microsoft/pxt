\documentclass[sigplan]{acmart}

  %\settopmatter{printfolios=true,printccs=false,printacmref=false}
  %\settopmatter{printfolios=false,printccs=false,printacmref=false}
  \usepackage{graphicx}
  \usepackage{listings}
  \usepackage{enumitem}

  % URL PACKAGE HACK
  \expandafter\def\expandafter\UrlBreaks\expandafter{\UrlBreaks%  save the current one
  \do\a\do\b\do\c\do\d\do\e\do\f\do\g\do\h\do\i\do\j%
  \do\k\do\l\do\m\do\n\do\o\do\p\do\q\do\r\do\s\do\t%
  \do\u\do\v\do\w\do\x\do\y\do\z\do\A\do\B\do\C\do\D%
  \do\E\do\F\do\G\do\H\do\I\do\J\do\K\do\L\do\M\do\N%
  \do\O\do\P\do\Q\do\R\do\S\do\T\do\U\do\V\do\W\do\X%
  \do\Y\do\Z}

  \newcommand{\MC}{MakeCode\ }
  \newcommand{\MCN}{MakeCode}
  \newcommand{\CO}{CODAL\ }
  \newcommand{\CON}{CODAL}
  \newcommand{\COLN}{codal}
  \newcommand{\UF}{UF2\ }
  \newcommand{\UFN}{UF2}
  \newcommand{\flameon}[1]{\emph{#1}}
  \def\dbhref#1#2{URL}
  %\def\dbhref#1#2{\href{#1}{#2}}
  \def\dburl#1{URL}
  %\def\dburl#1{\url{#1}}

  \setlist[itemize]{leftmargin=*}
  \lstset{ %
  language=C++,                % choose the language of the code
  basicstyle=\footnotesize,       % the size of the fonts that are used for the code
  numbers=left,                   % where to put the line-numbers
  numberstyle=\footnotesize,      % the size of the fonts that are used for the line-numbers
  stepnumber=1,                   % the step between two line-numbers. If it is 1 each line will be numbered
  numbersep=5pt,                  % how far the line-numbers are from the code
  backgroundcolor=\color{white},  % choose the background color. You must add \usepackage{color}
  showspaces=false,               % show spaces adding particular underscores
  showstringspaces=false,         % underline spaces within strings
  showtabs=false,                 % show tabs within strings adding particular underscores
  frame=single,           % adds a frame around the code
  tabsize=2,          % sets default tabsize to 2 spaces
  captionpos=b,           % sets the caption-position to bottom
  breaklines=true,        % sets automatic line breaking
  breakatwhitespace=false,    % sets if automatic breaks should only happen at whitespace
  escapeinside={\%*}{*)}          % if you want to add a comment within your code
  }

    %% For double-blind review submission, w/ CCS and ACM Reference
    %\documentclass[sigplan,10pt,review,anonymous]{acmart}\settopmatter{printfolios=true}
    %% For single-blind review submission, w/o CCS and ACM Reference (max submission space)
    %\documentclass[sigplan,10pt,review]{acmart}\settopmatter{printfolios=true,printccs=false,printacmref=false}
    %% For single-blind review submission, w/ CCS and ACM Reference
    %\documentclass[sigplan,10pt,review]{acmart}\settopmatter{printfolios=true}
    %% For final camera-ready submission, w/ required CCS and ACM Reference
    %\documentclass[sigplan,10pt]{acmart}\settopmatter{}


    %% Conference information
    %% Supplied to authors by publisher for camera-ready submission;
    %% use defaults for review submission.
    % \acmConference[PL'17]{ACM SIGPLAN Conference on Programming Languages}{January 01--03, 2017}{New York, NY, USA}
    % \acmYear{2017}
    % \acmISBN{} % \acmISBN{978-x-xxxx-xxxx-x/YY/MM}
    % \acmDOI{} % \acmDOI{10.1145/nnnnnnn.nnnnnnn}
    \startPage{1}

    %% Copyright information
    %% Supplied to authors (based on authors' rights management selection;
    %% see authors.acm.org) by publisher for camera-ready submission;
    %% use 'none' for review submission.
    \setcopyright{acmlicensed}
    %\setcopyright{acmcopyright}
    %\setcopyright{acmlicensed}
    %\setcopyright{rightsretained}
    %\copyrightyear{2017}           %% If different from \acmYear

    %% Bibliography style
    \bibliographystyle{ACM-Reference-Format}
    %% Citation style
    %\citestyle{acmauthoryear}  %% For author/year citations
    %\citestyle{acmnumeric}     %% For numeric citations
    %\setcitestyle{nosort}      %% With 'acmnumeric', to disable automatic
                                %% sorting of references within a single citation;
                                %% e.g., \cite{Smith99,Carpenter05,Baker12}
                                %% rendered as [14,5,2] rather than [2,5,14].
    %\setcitesyle{nocompress}   %% With 'acmnumeric', to disable automatic
                                %% compression of sequential references within a
                                %% single citation;
                                %% e.g., \cite{Baker12,Baker14,Baker16}
                                %% rendered as [2,3,4] rather than [2-4].


    %%%%%%%%%%%%%%%%%%%%%%%%%%%%%%%%%%%%%%%%%%%%%%%%%%%%%%%%%%%%%%%%%%%%%%
    %% Note: Authors migrating a paper from traditional SIGPLAN
    %% proceedings format to PACMPL format must update the
    %% '\documentclass' and topmatter commands above; see
    %% 'acmart-pacmpl-template.tex'.
    %%%%%%%%%%%%%%%%%%%%%%%%%%%%%%%%%%%%%%%%%%%%%%%%%%%%%%%%%%%%%%%%%%%%%%


    %% Some recommended packages.
    \usepackage{booktabs}   %% For formal tables:
                            %% http://ctan.org/pkg/booktabs
    \usepackage{subcaption} %% For complex figures with subfigures/subcaptions
                            %% http://ctan.org/pkg/subcaption

    \usepackage{courier}

    \begin{document}

    \acmConference[LCTES'18]{Languages, Compilers, and Tools for Embedded Systems}{June 19--20}{Philadelphia, PA, USA}

    %% Title information
    \title[Embedded Systems Programming for Education]{\MC and \CON: Intuitive and Efficient Embedded Systems Programming for Education}         %% [Short Title] is optional;
                                            %% when present, will be used in
                                            %% header instead of Full Title.
    %\subtitle{Microsoft MakeCode and Lancaster University Teams}                     %% \subtitle is optional


    %% Author information
    %% Contents and number of authors suppressed with 'anonymous'.
    %% Each author should be introduced by \author, followed by
    %% \authornote (optional), \orcid (optional), \affiliation, and
    %% \email.
    %% An author may have multiple affiliations and/or emails; repeat the
    %% appropriate command.
    %% Many elements are not rendered, but should be provided for metadata
    %% extraction tools.

    %% Author with single affiliation.
    \author{James Devine}
    \affiliation{
      \institution{Lancaster University, UK}            %% \institution is required
    }
    \email{j.devine@lancaster.ac.uk}

    \author{Joe Finney}
    \affiliation{
      \institution{Lancaster University, UK}            %% \institution is required
    }
    \email{j.finney@lancaster.ac.uk}

    \author{Micha\l\ Moskal}
    \affiliation{
      \institution{Microsoft, USA}            %% \institution is required
    }
    \email{mmoskal@microsoft.com}

    \author{Peli de Halleux}
    \affiliation{
      \institution{Microsoft, USA}            %% \institution is required
    }
    \email{jhalleux@microsoft.com}

    \author{Thomas Ball}
    \affiliation{
      \institution{Microsoft, USA}            %% \institution is required
    }
    \email{tball@microsoft.com}

    \author{Steve Hodges}
    \affiliation{
      \institution{Microsoft, UK}            %% \institution is required
    }
    \email{shodges@microsoft.com}


    %% Author with two affiliations and emails.
    % \author{First2 Last2}
    % \authornote{with author2 note}          %% \authornote is optional;
    %                                         %% can be repeated if necessary
    % \orcid{nnnn-nnnn-nnnn-nnnn}             %% \orcid is optional
    % \affiliation{
    %   \position{Position2a}
    %   \department{Department2a}             %% \department is recommended
    %   \institution{Institution2a}           %% \institution is required
    %   \streetaddress{Street2a Address2a}
    %   \city{City2a}
    %   \state{State2a}
    %   \postcode{Post-Code2a}
    %   \country{Country2a}                   %% \country is recommended
    % }
    % \email{first2.last2@inst2a.com}         %% \email is recommended


    %% Abstract
    %% Note: \begin{abstract}...\end{abstract} environment must come
    %% before \maketitle command
    \begin{abstract}
    % http://www.grandviewresearch.com/press-release/global-microcontroller-market
    Over the last decade, microcontrollers, the low-power low-cost workhorses of embedded systems, 
    are finding use in making and education. Furthermore, growth for microcontrollers is increasing due to
    demand for devices to monitor and control systems (e.g., Internet of Things, home automation).
    However, one generally needs a professional development environment 
    and substantial programming skills to develop applications for micrcontrollers. 
    
    We present a new open source platform for programming of microcontroller-based devices, with the goal
    of making it easy for \emph{anyone} to participate in creating with microcontrollers from 
    most \emph{anywhere} (a computer with a modern web browser and a USB port). 
    We evaluate the performance of the platform on devices ranging from the Arduino Uno
    to micro:bit and Adafruit Circuit Playground Express, popular boards used in making and creating
    around the world. We describe how the platform has been architected to make it easy to port
    to a wide range of microcontollers (\emph{anything}).

%    The platform supports visual block-based programming,
%    and JavaScript in a web app with editors, simulator and complete compile chain that compiles 
%    user code to binary in the browser. Static TypeScript, a subset of the TypeScript languge, 
%    mediates between worlds of JavaScript and the C++ runtime for the microcontroller, 
%    which is precompiled and cached in the web app. Both the JavaScript and C++ runtimes support
%    an event-based programming model with support for non-preemptive fibers provides a simple starting 
%    point for beginners, and progression to scenarios which benefit from concurrency. 
\end{abstract}

    %% Keywords
    %% comma separated list
    %\keywords{keyword1, keyword2, keyword3}  %% \keywords are mandatory in final camera-ready submission


    %% \maketitle
    %% Note: \maketitle command must come after title commands, author
    %% commands, abstract environment, Computing Classification System
    %% environment and commands, and keywords command.
    \maketitle

    \renewcommand{\shortauthors}{J. Devine, J. Finney, M. Moskal, P. Halleux, T. Ball, S. Hodges}

    \section{Introduction}
\label{sec:intro}

Over the last decade, microcontrollers, the workhorses of embedded systems, have become
central to efforts in making~\cite{dougherty2012maker} and education. For example, the Arduino project
(\url{www.arduino.cc})~\cite{buildingArduino2014},
started in 2003, created the Uno board using an 8-bit Atmel
AVR microcontroller. The Uno makes its microcontroller's I/O pins available via headers;
external hardware modules (shields) may be connected to these headers to extend
the Uno's capability. The Arduino ecosystem has grown tremendously in the past 15 years,
with the support of companies such as Adafruit Industries (\url{www.adafruit.com}) and
Sparkfun Electronics (\url{www.sparkfun.com}), who resell Arduino and make their
own Arduino-compatible boards. 

The Arduino platform has the following characteristics, common to many programming
environments for microcontrollers~\cite{XYZ}:
\begin{itemize}
\item it uses C/C++ as the starting programming language;
\item it loads code using 1980's era bootloader technology;
\item it encourages polling of sensors;
\item it lacks many interactive features of modern IDEs;
\end{itemize}
These characteristics make such systems non-trivial for beginners to work with, 
require the installation of OS-specific drivers/applications/toolchains,
and leads to poor programming practices.

The Java language (among others) held out the promise of a better way forward for 
programming microcontrollers, but XYZ.  
\flameon{we need more description of why current ways of programming microcontrollers 
make for a high barrier to entry; also need to take on Java head on here, as well as
RTOS and MicroPython.}

In contrast to the situation for programming microcontrollers, 
on the web we find many excellent environments for introductory programming.
Visual block editors such as Scratch (\url{https://scratch.mit.edu/})~\cite{ScratchCACM2009,BlocksBeyondCACM2017}
and Blockly (\url{https://developers.google.com/blockly/})~\cite{Blocky2015}
allow the creation of programs without the possibility of syntax errors.
The programming models associated with Scratch and Blockly generally are
event based, freeing the programmer from the need to poll.
HTML, CSS and JavaScript allow a complete programming experience to be delivered as an interactive
web app, including editing with intellisense, code execution and debugging~\cite{Monaco}. 

With a surge in the demand of microcontroller-based devices for education~\cite{XYZ}, 
there is a need to simplify the programming of such devices so that they suitable 
for novice users in restricted environments.
Therefore, we have created a new programming platform that bridges the worlds of 
the microcontroller and the web app. 

The major goals of the platform are to:
% TypeScript / Blocks + MakeCode
(1) make it simple to program microcontrollers in a higher-level language,
using nothing more than a web app;
% TypeScript and Blocks prevent users from making boo boos.
(2) provide a safe environment for users to develop programs for microcontrollers;
% simulator, auto completion...
(3) create a feature rich and extensible development environment that decreases time taken to program a microcontroller (time to awesome);
% UF2 is awesome
(4) allow a users' compiled program to be easily installed on a microcontroller;


The platform consists of a stack of four novel technologies, the subject of
this paper:
\begin{itemize}
\item \emph{\MC (\href{https://makecode.com}{makecode.com})}, a web app that supports both visual block programming and text programming,
via \emph{Static TypeScript}, with conversion between the two program representations (Section~\ref{sec:makecode});

\item \emph{Static TypeScript}, a statically-typed subset of TypeScript (\url{www.typescriptlang.org}),
a gradually-typed superset of JavaScript, for fast execution on low-memory devices, with
a simple model for linking against pre-compiled C++ (Section~\ref{sec:sts});

\item \emph{\CO (the Component-oriented Device Abstraction Layer)}, an event driven, multi-threaded, C++ runtime environment that bridges the semantic gap between higher-level languages and the hardware,
modelling each hardware component as a software component (Section~\ref{sec:codal});

\item \emph{USB Flashing Format} (UF2), a new file format designed for flashing microcontrollers 
over the Mass Storage Class protocol (USB pen drives); the format greatly speeds the installation of user
programs and is robust to differences in operating systems (Section~\ref{sec:uf2}).
\end{itemize}
The \MC web app is the entry point of the platform, and has in-browser execution via a device simulator, as well as compilation to machine code and linking against a
pre-compiled C++ runtime (\emph{\CON}). No C/C++ compiler is invoked to compile user code and the result of compilation is a binary file that is ``downloaded'' from the web app to the user's
computer and then flashed to the microcontroller (exposed as a USB pen drive) 
via a simple file copy operation,  with the aid of the \emph{UF2} file format and supporting firmware. 


These four advances enable beginners to get started programming microcontrollers from 
any modern web browser, and enable hardware vendors to innovate and safely add new 
components to the mix using Static TypeScript, leveraging its
foreign function interface to C++.
Once the web app has been loaded, all the above functionality works offline 
(i.e., if the host machine loses its connection
to the internet). 

\begin{table}[]
\centering
\begin{tabular}{|l|r|r|r|r|r|}
\hline
            &          &            & \bf{Word} &          &             \\
\bf{Device} & \bf{RAM} & \bf{Flash} & \bf{Size} & \bf{CPU} & \bf{Chip}   \\ \hline
Uno         & 2 kB     & 32 kB      & 8         & AVR      & ATmega328P  \\ \hline
micro:bit   & 16 kB    & 256 kB     & 32        & M0       & nRF52       \\ \hline
CPX         & 32 kB    & 256 kB     & 32        & M0       & SAMD21      \\ \hline
\end{tabular}
\caption{\label{table:devices}A subset of devices supported by the platform. 
CPX is Adafruit's Circuit Playground Express. M0 denotes Cortex-M0.}
\end{table}
      
Table~\ref{table:devices} lists three of the devices supported by our platform, ranging
from the highly resource-contrained Arduino Uno to the slightly less constrained space of
the micro:bit and Adafruit Circuit Playground Express (CPX).


\subsection{Running Example}

Figure~\ref{fig:example} shows a program in the Static
TypeScript subset that implements a simple ``stopwatch'' timer
for the micro:bit.
The program has three top-level statements:
the first initializes the global variable \emph{start} (line 1); the
second registers an event handler (a lambda function) to execute
each time button A of the micro:bit is pressed (line 3); the
third registers a lambda function to run forever on a fiber (line 18),
to animate the micro:bit's 5x5 LED screen whenever the timer is active. 

Note that this program is a JavaScript program, as there are no 
types mentioned explicitly. However, all the functions called in
this program are part of the runtime and are explicitly
typed.  As a result, the static type of every variable and expression
can be inferred by TypeScript's type inference.

The program also shows off the use of the non-preemptive concurrency
model supported by both \MC (for JavaScript) and \CO (for C++). 
The fiber running the forever statement executes the lambda inside a ``while (true)'' 
loop that yields (via a call to \emph{basic.pause}) after each call to the lambda.
This gives the button-press event handler a chance to execute
upon user input (in a separate fiber). Although the global variable \emph{start} is 
shared by the two fibers, there is no data race due to the non-preemptive 
scheduling model. 

\flameon{TODO: event queueing/execution model???}



\begin{figure}
\begin{lstlisting}
let start = 0

input.onButtonPressed(Button.A, () => {
  if (start == 0) {
    start = input.runningTime()
  } else {
    let d = input.runningTime() - start
    start = 0 
    basic.clearScreen()
    basic.pause(1000)
    basic.showString(d/1000 + "." + d%1000)
  }
})

basic.forever(() => {
  if (start) {
    led.toggle(Math.random(5), Math.random(5))
  }
})
\end{lstlisting}
\caption{\label{fig:example}Running example.}
\end{figure}

\subsection{Evaluation}

%we encourage the reader to choose a target
%from \url{www.makecode.com} and experiment with programming it, to appreciate the
%qualitative aspects of the platform, namely its simplicity and ease of use.
In this paper, we evaluate quantitative aspects of the platform
with respect to the devices from Table~\ref{table:devices}. In particular, we
consider:
\begin{itemize}
\item the time to compile Static TypeScript user code (to machine code) with respect
      to the GCC C/C++ toolchain, as well as the size of the resulting executable;
\item the time to load code onto a microcontroller using UF2, compared to standard bootloaders
      such as Arduino and ARM's DAPlink;
\item the performance of a set of small benchmarks, written in both Static TypeScript and C++,
      compiled with the \MC and GCC toolchains;
\item \emph{energy consumption: \CO vs. Arduino}
\item \emph{native code vs. bytecode} we
      evaluate memory consumption and code performance for native code generation
      vs. bytecode generation and interpretation on the Uno.
\end{itemize}

\flameon{TODO: we should present some of high-level experimental results here.}

All of the platform's components are open source on GitHub.
  
Sections~\ref{sec:makecode} to~\ref{sec:uf2} presents the four major components of the platform, top-down,
as referenced before. Section~\ref{sec:evaluate} evaluates the performance of the platform,
Section~\ref{sec:related} discusses related work, and Section~\ref{sec:conclude}
concludes.

    \section{Architecture}

\begin{figure}[t]
    \includegraphics[width=4.8in]{makecodeFig.pdf}
\caption{\label{fig:makecode}\MC web app}
\end{figure}

Currently, due to the low-level nature of MCU programming, additional drivers and software are commonly required to be locally installed in order to program them. In restricted environments, like libraries and schools, safeguards are put in place to protect users and machines from downloading or installing additional software. Further, Internet connections are unreliable in remote locations. These factors create barriers for designing a programming environment suited to a diverse audience. Therefore, a solution is required that can: (1) operate in restricted environments; (2) operate with little, or no internet connectivity; and (3) work on any operating system.

To meet these requirements, we created the \MC web app (Figure~\ref{fig:makecode}), the entry point of the platform. The web app can be accessed from any modern web browser and cached locally for \emph{entirely offline use}. The \MC web app incorporates the open source Blockly (\emph{\href{https://github.com/google/blockly}{blockly}}) and Monaco (\emph{\href{https://github.com/Microsoft/monaco-editor}{monaco-editor}}) editors (upper-left), an in-browser device simulator (upper-right) for testing programs before transferring them to the physical device, as well as \emph{in-browser compilation} of Static TypeScript to machine code and linking against the C++ runtime (\emph{\CON}), pre-compiled (by a cloud service, lower left).

%The statically-typed runtime and type inference on user code allows users to write code that looks like plain JavaScript (see Figure~\ref{fig:example}).

\MC devices appear as USB pen drives when plugged into a computer. After a user has finished developing a program, the compiled binary is ``downloaded'' locally to the users computer and then transferred (flashed) to the MCU (exposed as USB pen drive) by a simple file copy operation. No additional installation is required to program the MCU as drivers for pen drive come pre-installed on many operating systems (MacOS, Windows, Linux, Android, ChromeOS).

These advances enable beginners to get started programming MCUs from any modern web browser, and offer a safe environment for hardware vendors to innovate and add new components using Static TypeScript. All of the platform's components are open source on GitHub (links removed).
    \section{\MC: Design and Implementation}
\label{sec:makecode}

Figure~\ref{fig:screenSnap} shows a screen snapshot of the \MC web app for the micro:bit
with the main parts labelled: 
(A) the menu bar allows switching between views of blocks and JavaScript;
(B) the simulator provides feedback on user code executed in the browser;
(C) the toolbar provides access to device-specific APIs and programming elements;
(D) the programming canvas; 
(E) the download button invokes the compiler to produce a binary executable.

The \MC web app encapsulates all the components needed to deliver a programming experience 
for microcontroller based devices, free of the need for a C++ compiler for the compilation of user 
code.
The web app is written in TypeScript and also incorporates the TypeScript compiler and 
language service as well. 
The app is built from a target (e.g., \emph{\href{https://github.com/microsoft/pxt-microbit}{pxt-microbit}})
which extends the core framework (\emph{\href{https://github.com/microsoft/pxt}{pxt}}).
The remaining subsections describe the parts of Figure~\ref{fig:makecode}, 
which shows the high-level components of the web app.

\begin{figure*}[t]
    \includegraphics[width=5.5in]{makecodeFig.pdf}
\caption{\label{fig:makecode}\MC Architecture}
\end{figure*}

\subsection{Device Runtime and Shim Generation}

An \MC target is defined, in part, by its device runtime, which can be a combination of C++ 
and Static TypeScript (STS) code. The C++ runtime for the target microcontroller is precompiled 
and stored in the cloud. The runtime binary also is stored in the HTML5 application cache (with 
other assets) so that the web app can function when the browser is offline. Additional runtime
components may be authored in STS, which allows the device runtime to be updated without the
use of C++, and permits components of the device runtime to be shared by both the device
and simulator runtimes. The ability to author the device runtime in both STS and C++ is
a unique aspect of \MC's design.

Whether runtime components are authored in C++ or STS, all runtime APIs are exposed as fully-typed
TypeScript definitions, as described later. A full-typed runtime improves the end-user experience 
by making it easier to discover APIs; it also enables the type inference provided by the TypeScript 
compiler to infer types for (unannotated) user JavaScript programs.

\MC supports a simple foreign function interface from TypeScript to C++ based on namespaces,
enumerations, functions, and basic type mappings. \MC uses top-level namespaces (in both C++ and
TypeScript) to organize sets of related functions.  Preceding a C++ namespace, enumeration, or function
with \emph{//\%} indicates that \MC should map the C++ construct to TypeScript.
Within the \emph{//\%} comment, attributes are used to define the visual appearance for that
language construct, such as for the LED namespace in the micro:bit target:

\begin{lstlisting}
//% color=#5C2D91 weight=97 icon="\uf205"
namespace led { 
...
\end{lstlisting}

Figure~\ref{fig:screenSnap}(C) shows the toolbox of API and language categories, where the LED
namespace can been seen. Here is the C++ file defining the micro:bit's led namespace and its functions:
~\href{https://github.com/Microsoft/pxt-microbit/blob/master/libs/core/led.cpp}{pxt-microbit/libs/core/led.cpp}.

Mapping of functions and enumerations between C++ and TypeScript is straightforward
and performed automatically by \MC. 
Here's an example of the C++ function plot in the led namespace that wraps a more
complex function call of the underlying device runtime to set/plot an LED in the micro:bit display:

\begin{lstlisting}
//% blockId=device_plot block="plot|x %x|y %y"
//% x.min=0 x.max=4 y.min=0 y.max=4
void plot(int x, int y) {
    uBit.display.image.setPixelValue(x, y, 1);
}
\end{lstlisting}

We'll describe the attribute definitions in the \emph{//\%} comment in the next section. 
\MC uses a TypeScript declaration file to describe the TypeScript elements corresponding
to C++ namespaces, enumerations and functions.  We call such files shim files.
Since the C++ plot function is preceded by a \emph{//\%} comment, 
\MC adds the following TypeScript declaration to the shim file (shims.d.ts) and copies
over the attribute definitions in the comment. \MC also adds an attribute definition mapping
the TypeScript shim to its C++ function (shim=led::plot):

\begin{lstlisting}
//% blockId=device_plot block="plot|x %x|y %y
//% x.min=0 x.max=4 y.min=0 y.max=4 shim=led::plot
function plot(x: number, y: number): void;
\end{lstlisting}

Here is the shim file generated from C++ micro:bit device runtime:
\href{https://github.com/Microsoft/pxt-microbit/blob/master/libs/core/shims.d.ts}{pxt-microbit/libs/core/shims.d.ts}.

To support the foreign function interface, \MC defines a mapping between C++ and TypeScript types.
Boolean and void have straightforward mappings from C++ to TypeScript (bool $\rightarrow$ boolean, void $\rightarrow$ void). 
As JavaScript only supports number, which is a C++ float/double, \MC uses TypeScript's support
for type aliases to name the various C++ integer types commonly used for microcontroller programming
(int32, uint32, int16, uint16, int8, uint8). 
% don't use int, unsigned etc. they are actually different sizes on different compilers for AVR
This is particularly useful for saving space on 8-bit architectures such as the AVR. 

\MC allows a set of C++ functions with the same first parameter (of type Foo) to be
exposed as a TypeScript interface Foo as follows: this set of C++ functions must be grouped
inside a namespace of the name FooMethods.  See, for example, how a C++ buffer abstraction is exposed:
\href{https://github.com/Microsoft/pxt-microbit/blob/master/libs/core/buffer.cpp}{pxt-microbit/libs/core/buffer.cpp}.
You can find the resulting TypeScript Buffer interface in the previously referenced shim file for the micro:bit.

\MC includes reference counted C++ types for strings, lambdas (with
up to three arguments and a return type) and collections.  
\MC does not yet include garbage collection, so advanced users who create cyclic
data structures must be careful to break cycles in order to ensure complete deallocation. 

\subsection{Block Metadata Generation}

Both C++ and TypeScript APIs can be specially annotated (minimally via 
\emph{//\% block}) so that the \MC compiler generates the needed
Blockly metadata to expose an API as a visual block. So, to expose the previously
encountered plot function as a visual block (as well as a TypeScript function), one simply needs:
\begin{lstlisting}
//% block
void plot(int x, int y) { . . . }
\end{lstlisting}

Additional attribute definitions can provide text descriptions for the block, project function
parameters (thus simplifying the API available via Blockly), and describe other visual/functional
characteristics of the block.  \MC uses the types of function parameters to select appropriate
Blockly widgets.  For example, an enumeration is represented by a dropdown menu in blocks.
For more information on the block-specific annotations, see 
\url{https://makecode.com/defining-blocks}. 
\MC's support for Blockly means that for the common case, the target developer doesn't need
to know anything about the Blockly framework.  For more sophisticated needs, one can directly access
the Blockly framework. 

\subsection{Editors and Code Conversion}

\MC uses the Blockly (\emph{\href{https://github.com/google/blockly}{blockly}}) and Monaco 
(\emph{\href{https://github.com/Microsoft/monaco-editor}{monaco-editor}}) editors to allow the user to code with
visual blocks or TypeScript. The editing experience is parameterized by the full-typed device
runtime, which provides a set of categorized APIs to the end-user, based on namespaces, as
previously described. These APIs are visible in both editors via a toolbox to the immediate
left of the programming area. The Blockly and Monaco toolboxes show the same set of APIs, to
help in transition from coding with blocks to coding with JavaScript. More advanced TypeScript
APIs can be discovered in Monaco via code intellisense.

The Blockly program representation is compiled to Static TypeScript in a syntax-directed manner
(see \emph{\url{https://github.com/Microsoft/pxt/tree/master/pxtblocks}{pxtblocks}}). A key issue is the need for
type inference on the Blockly representation, as variables generally are defined and used without
being declared in Blockly. \MC uses a simple unification type inference to assign a
unique type to each variable.  
% this may not be possible:
%In the future, we expect to use TypeScript's type inference instead
%and eliminate the need for separate type inference over the Blockly representation. 
TypeScript supports programming constructs that are not available in Blockly, such as classes.
Such constructs are converted into grey uneditable blocks in Blockly, with the construct's program
text intact. This means \MC always can decompile a TypeScript program to Blockly and then recover
the program text of the grey blocks when converting from Blockly back to TypeScript
 (see \emph{\href{https://github.com/Microsoft/pxt/blob/master/pxtcompiler/emitter/decompiler.ts}{decompiler.ts}}). 

\subsection{Compilation Pipeline}

\MC first invokes the TypeScript language service to perform type inference and type checking on the 
user's program, using the TypeScript declaration files for the device runtime.   It then checks that the
user's program is within the STS subset through additional syntactic and type checks over the adorned
abstract syntax tree (AST) produced by the language service (detailed in Section XYZ).  Assuming all the
above checks pass, \MC then performs tree shaking and compilation of the AST of the user code and
device runtime to an intermediate representation (IR) that makes explicit: labelled control flow among a
sequence of instructions with conditional and unconditional jumps; heap cells; field accesses; store operations,
and reference counting.

There are four backends for code generation from the IR. The first backend simply generates JavaScript,
for execution against the simulator runtime.  A second backend generates assembler, parameterized by a
processor description.  Currently supported processors include ARM's Cortex class (Thumb instructions)
and Atmel's Atmega class (AVR instructions). A separate assembler, also parameterized by an instruction
encoder/decoder, generates machine code and resolves runtime references, producing a binary executable.
A third backend generates bytecode instructions and a fourth backend targets C\#. 
The compiler chain
can be found at \emph{\href{https://github.com/Microsoft/pxt/tree/master/pxtcompiler/emitter}{pxtcompiler/emitter}} and 
\emph{\href{https://github.com/Microsoft/pxt/tree/master/pxtlib/emitter}{pxtlib/emitter}}.


\subsubsection{Asynchronous Functions}

An important part of the compilation process is to allow users to call asynchronous 
TypeScript functions (identified through the \emph{//\% async} annotation) 
as if they were blocking functions.  
For execution in the browser,
every function is compiled so that it can be suspended (at the return of a call) and later resumed (at the same point). 
The default behavior at a suspension point is to immediately resume execution.  For a call to an async function,
the default behavior is overridden by the compiler, which suspends execution of the current function. 
Upon completion of the async function call, the current function then is resumed. Such async functions are written
by runtime developers, not end-users, and greatly simplify the JavaScript
programming model for end-users. For example, although the JavaScript simulator runtime uses promises to 
achieve asynchronous execution in a single-threaded context, these promises are hidden from the end user. 
The CODAL device runtime supports fibers with the ability to pause, so for compilation to a device, 
the compiler simply emits a call to pause at a suspension point. 

\emph{TODO: need a small example to clarify how it all works}

% tball: I added TypeScript above to make it clear 
%\emph{TODO: The async annotation is only relevant for the JS code. In ARM/AVR it doesn't do anything.
%One way to say it is to say it's there to simulate cooperative multi-threading in JS, since
%it is already implemented in C++ by CODAL.}

\subsubsection{Untagged and Tagged Strategies}

The \MC compiler supports the Static TypeScript language subset described in Section~\ref{sec:sts},
with two compilation strategies: untagged and tagged. Under the untagged strategy,
a JavaScript number is interpreted as a C++ int by default and the type system is used
to statically distinguish primitive values from boxed values. As a result, the untagged
strategy is not fully faithful to JavaScript semantics: there is no support for floating
point and the Null and Undefined types are represented by the default integer value of zero.
This strategy is used for the micro:bit and Arduino Uno targets. 

In the tagged strategy, numbers are either tagged 31-bit signed integers, or if they do not fit, 
boxed doubles. Special constants like false, null and undefined are given special values 
and can be distinguished. The tagged execution strategy has the capability to fully support
JavaScript semantics. This strategy is used for all SAMD21 targets.

\subsection{Simulator}

A \MC target can provide an alternate TypeScript implementation for each API in the device runtime, for use in the device
simulator. As this code runs in the web browser (not on the actual device) and manipulates the DOM, the developer is free to
use all of TypeScript/JavaScript's features. (As an aside, \MC also support ``simulator-only'' targets that have no 
associated device; in these cases, the ``device runtime'' is defined solely by the simulator APIs.) 

The simulator allows the user to experience the basic functions of the device in the browser and to test their code
before deploying it to the actual device. The simulator has proxy widgets for sensors such as accelerometer (mouse motion),
temperature and light, allowing the user to control the sensor's value.  The simulator only provides basic functionality
and is far from a complete device emulation.   For example, it is not possible for the user to simultaneously modify two
inputs to the simulated device, while it is possible with the actual device (i.e., shaking it to change the accelerometer
reading while pushing one of the device's buttons).

%\MC provides various components to make device simulators easier to build: board, parts, wiring, etc.

\subsection{Packages and Custom Editors}

Packages are \MC's dynamic/static library mechanism for extending a target (by adding new code/data to the device
and simulator runtimes, as well as accompanying documentation). The following package extends the micro:bit target so
that the micro:bit can drive a NeoPixel strip of RGB LEDs: \url{https://github.com/Microsoft/pxt-neopixel}. 

To see how this
package is surfaced to the end-user, visit \url{http://makecode.microbit.org/} and select the ``Add Package'' option from the
gear menu; you will see the package ``neopixel'' listed in the available options. If you click on it, a new block category
named ``Neopixel'' will be added to the editor. In this scenario, PXT dynamically loads the (white listed) neopixel 
package directly from GitHub, compiles it and incorporates it into the web app. Packages also can be bundled with a web
app (the analog of static linking).  

For packages containing C++ files, a new C++ runtime has to be compiled in the cloud.
This is achieved by collecting all C++ files (in all packages plus the CODAL runtime),
computing a hash, checking in the local in-browser cache if such a runtime was retrieved
before, and otherwise asking the cloud service to compile it.
Of course, the cloud service will return a cached copy, if the same set of C++
files was compiled before. The cache hit rates, in both the browser cache
and the cloud cache, are very high. 
The first hit rate is high because the cloud needs to be consulted
only when a new package is added to the project, and this particular combination 
of packages was never used before on current machine.
The second hit rate is high because people rarely combine many packages (due to
hardware and memory constraints), and there are only so many of them.

    \begin{figure}[t]
    \includegraphics[width=4.5in]{codalFig.pdf}
    \caption{\label{fig:codal}Detailed relationships between CODAL repos.}
\end{figure}

\section{The CODAL Runtime}
\label{sec:codal}

Typically written in C and/or C++, platforms for microcontroller programming all share a common design goal: to support developers by providing primitives and programming abstractions. Platforms can range from simple C++ classes that control hardware, to real-time operating systems (RTOSs) with scheduling and memory management.

The Arduino ecosystem is an example of a simple platform where the developer uses high-level APIs to control hardware; there is no scheduler, and memory management is discouraged through a heavy emphasis on global variables.  The Arduino programming model is based on polling: an Arduino ``sketch'' is a program template that consists of a setup procedure, for initialization of data structures, and a loop procedure; programmers implement the body of \textit{setup} and \textit{loop}, where they explicitly poll the state of the sensors or the microcontroller's pins.

At the other end of the spectrum is mbed OS, which is aimed at developers who are familiar with non-preemptive scheduling, memory management, and condition synchronisation; a more complex environment. A number of high-level drivers are provided and the programming model is determined by the developer, commonly event or interrupt driven.

Although platforms like Arduino and mbed OS are widely used by C/C++ developers, higher-level languages opt for virtual machines to execute on microcontrollers as neither extreme aligns with the programming models seen in higher level languages: mbed OS is pre-emptive, JavaScript is not; Arduino is based on monolithic polling. There is therefore a need to develop a layer that: (1) bridges the semantic gap between the higher-level language and the hardware; (2) does so in a memory, energy and instruction efficient way; and (3) supports a number of platforms with various resource contraints and capabilities.

\subsection{Codal}

The component-oriented device abstraction layer (CODAL) aims to provide a layer that is multi-platform, makes efficient use of memory and energy, provides both synchronous and asynchronous API variants, and offers a good experience for native C and higher level languages.

CODAL provides a set of components that abstract away microcontroller specifics, a non-preemptive scheduler that minimises the need for resource locking primitives whilst allowing asynchronous operation, and an eventing subsystem common to all CODAL Components that enables the decoupling of system and user code. A CODAL component represents software or hardware drivers (an I2C device, a GPIO pin, a Message Bus, a Bluetooth device); any component can generate or consume events. CODAL supports both polling and asynchronous (event-driven) programming models, which enables higher level languages to map straight onto native C/C++ APIs.

Unlike mbed, which is designed to make all devices look programmatically alike, CODAL enables devices with additional on-chip hardware to specialise device drivers with APIs that can then be used by the end application developer.

% more required here, needs to be reflowed...

\subsubsection{Structural Overview}

* a nice diagram
* an object model that factors out common code, but pushes platform specific code.
* why it's required:
    - platform independence
    - extensibility
    - platform specialisation
* explanation
Worked example of how everything fits together, short, only a few sentences.

* the core
* the targets
* supporting library

\subsubsection{Message Bus and Events}

Message passing via events is a common technique and is the foundation of many operating systems --- yet in most systems events are static used only to notify users of key events within the system (e.g. SIGKILL on UNIX).

In CODAL, we offer extensible events where the data contained in an event can be arbitrary, commonly application specific. Application developers can then ``listen'' to events, specifying an id (namespace), a value (unique within a namespace), and a function to be invoked when an event is raised. Events pass through the message bus, if a corresponding listener is in place, the function indicated by the listener is invoked. The eventing model aligns well to languages like JavaScript as well as Scratch blocks such as ``on button pressed''. An example is given below:

\begin{lstlisting}
#include "CircuitPlayground.h"
CircuitPlayground cplay;

void a_listener() { // do something }

int main() {
    cplay.messageBus.listen(100,100,a_listener);
    Event(100,100);
}
\end{lstlisting}

Not only do events enable easy modularisation of code, but they also safeguard the application programmer from situations where incorrect code could cause unexpected behaviour. Take for example an application to detect a button press, there are usually two solutions: (1) poll the state of a pin; or (2) configure an interrupt to fire when the state of a pin changes. Taking the latter approach, it could look something like this:

\begin{lstlisting}
int state = 0;

void buttonClicked() {
    int gpioState = PIN0;
    // button released, blink LED!
    if(state == 1 && gpioState == 0) {
        set_state(LED, 1);
        wait_ms(1000);
        set_state(LED, 0);
    }
    state = gpioState;
}

int main() {
    configure_interrupt(PIN0, buttonClicked);
    while(1);
}
\end{lstlisting}

The above code sample is considered bad practice as it prevents other interrupts (like Timers) from firing for at least a second when a button is released, however, this is often the first application that a user will create. Such frameworks advise users to avoid blocking functions and waiting in interrupt context, thus sidestepping the problem by punting responsibility to the user --- CODAL uses the message bus abstraction to shield users from such scenarios:

\begin{lstlisting}
#include "CircuitPlayground.h"
CircuitPlayground cplay;

void buttonClicked() {
    cplay.io.led.setDigitalValue(1);
    cplay.sleep(1000);
    cplay.io.led.setDigitalValue(0);
}

int main() {
    cplay.messageBus.listen(ID_BUTTON_A,DEVICE_BUTTON_EVT_CLICK,buttonClicked);

    while(1) cplay.sleep(100);
}
\end{lstlisting}

Note that the user doesn't have to configure any interrupts, as they are handled by pre-existing drivers.

% Stuff to add (it's already quite long):
% * Provides a similar mechanism seen in higher level languages
% * Timer and queues?
% * message passing microkernel

\subsubsection{Fiber Scheduler}

Novice users often want to perform multiple operations concurrently, generally requiring threads. Threading is a confusing concept: users have to learn about locks, semaphores, and preemption, to prevent race conditions. Threads can also consume a large number of resources depending on the model that is adopted by the runtime environment.

CODAL takes a non-preemptive scheduling approach which reduces the need for resource locking primitives. In CODAL's scheduling model, fibers (a lightweight thread) are RAM efficient and are only paged out when a user explicitly calls ``sleep'', meaning that the stack depth at the point where a fiber is paged out is small. Fibers have a variable stack size, and when paged out the entire stack is duplicated to the heap --- this would normally be ill-advised, but due to a usually small stack size, this is more efficient than approaches where there is a mandated stack size.

Events and the message bus are integral to CODAL even extending to fibers, which can block and wait on events to complete:

\begin{lstlisting}
#include "CircuitPlayground.h"
CircuitPlayground cplay;

void block() {
    // block the calling thread until the user clicks button a!
    fiber_wake_on_event(ID_BUTTON_A,DEVICE_BUTTON_EVT_CLICK,buttonClicked);
    cplay.io.led.setDigitalValue(1);
}

int main() {
    create_fiber(block);
    // wait until the button
    fiber_wait_for_event(ID_BUTTON_A,DEVICE_BUTTON_EVT_CLICK,buttonClicked);
    cplay.io.led.setDigitalValue(0);
}
\end{lstlisting}

\subsubsection{Drivers}

Drivers are commonly low level interfaces that control external or integrated hardware on a device. For embedded developers, creating and using drivers involves translating datasheets into program code and using low-level interface, which can alienate novice programmers.

In CODAL, drivers abstract away the complexities of the underlying hardware into reusable, extensible, easy-to-use components; for every hardware component there is a software component. There are three types of drivers in CODAL: (1) an abstract specification of a driver model common to most devices (e.g. I2C, Serial); (2) a driver that relies only on the interfaces specified in a driver model  (e.g. an I2C based driver); and (3) the concrete implementation of the abstract driver model, which may be platform specific or platform agnostic. Not all devices are the same, and by generalising the interface, devices can introduce platform specific optimisations and specialisations.

Interrupts are integral to drivers: in Serial communications it is useful to know when a byte has been sent or received. As illustrated in the button example, performing operations in interrupt context can be detrimental to the device. CODAL drivers use events to decouple computational tasks, that may take a variable length of time, from interrupt context.

% * motivation
% * example - I2C, microphone (DMA)
% * one component per piece of hardware or software
% * Provides a common interface
% * uses events to decouple from interrupt context.

\subsubsection{Memory Management}

Standard libraries in C++ do not offer reentrant versions of memory allocation calls like ``malloc'' and ``new'', critically this forbids the allocation of memory in interrupt context.

CODAL implements its own heap allocater that is designed to safeguard users from the scenario above, introducing reentrant versions of malloc. The heap allocater is flexible, reconfigurable, and repurposable, allowing the specification of multiple heaps across memory and it is optimised for common sizes and repeat allocations.

CODAL has a number of managed types that use reference counting to determine when memory should be freed. Internally, managed types use malloc to copy stack allocated data to the heap --- the reentrant behaviour of malloc and free are key here, as references can be created and destroyed regardless of processor context.
% * provides an interrupt safe environment for memory allocation
% * flexible implementation
%     - multiple heaps
%     - reconfigurable, repurposeable
% * optimisations for common sizes and repeated patterns
% * edu
% * couple of sentences on types
%     - ref counted, malloced types.

\subsubsection{Streams}

% * motivation
% * example
% * Pull and Push model
% * Adopted by all stream capable interfaces, from Serial to I2C
% * Worked example, audio recog, playback

\subsubsection{A code sample}

% * That says this is awesome!
% * how it fits together...

\subsection{Evaluation}

% - Profile fibers, how much do they actually use? are they any better?
%     - average RAM consumption for X
%     - Stack is typically small...
%     - types aid the stack size (heap allocated)
% - Compare memory allocator to lib c? (probably not that interesting)
% - benefits of componentisation?
%     - reusability, extensibility
% - How well does \MC + codal perform on each device? (SAMD, ATMEGA, NRF52)
%     - \MC + codal, native codal, mbed.
%     - cost benefit analysis of each.
%     - memory (flash and RAM) consumption
%     - CPU cycles
%         - context switch
%         - cost of events
%         - code gen, compare against micropython / espruino
%     - energy efficiency
%         - environmental sensing across 3 different platforms, mbed, espruino, codal
    \section{USB Flashing Format}
\label{sec:uf2}

There are several ways of installing a user program in the non-volatile flash memory
of a microcontroller. In professional scenarios one typically uses a debugger
interface of the target, with a specialized debugger chip and a native application
talking to it. Hobbyist typically employ a custom protocol over RS-232 serial
line. This typically requires operating system drivers and a native application.

ARM mbed uses DAPLink firmware, which presents itself to an external computer
as a USB pen drive. No special drivers need to be installed, as operating
systems support pen drives out of the box. The USB pen drive protocol (USB Mass
Storage Class or MSC) is a block-level protocol (generally 512 bytes) with no
concept of files. 

DAPLink exposes a small virtual FAT file system, which
never changes due to OS writes - it contains an informational HTML file, but
the file allocation table and the root directory are otherwise empty.
When the OS tries to read a block, DAPLink computes what should be there, 
based on compiled-in contents of the HTML file.
On file system writes, DAPLink detects the Intel HEX format~\cite{IntelHEX}, 
decodes it and flashes the file's contents into the target microcontroller's memory. 
Other writes are ignored.

DAPLink needs to implement heuristics to deal with quirks of FAT file
system implementations in various operating systems (order of writes, various meta-data files
that are created and need to be ignored, etc.). It is acceptable since DAPLink runs
on a separate debugger chip, however in a single-chip setup it is quite heavy for a bootloader occupying 
flash space of a target MCU.

Instead of dealing with these quirks we decided to keep the virtual FAT architecture, but change 
the file format, so that MSC writes can be easily acted upon.
Flashing Format (\UF) files consist of one or more 512-byte self-contained blocks.
The blocks have magic numbers, the payload data to be written to flash,
and the address where it should be written.
Thus, on every 512-byte write via the USB MSC, the bootloader can quickly and easily
check if the block being written is part of a \UF file (by comparing magic numbers)
and if so, write it immediately. In fact, this can be implemented in a streaming
fashion with less than 100 bytes of RAM, as on the ATmega16u2 interface chip
on Arduino UNO.
The minimal implementation of the \UF bootloader is 1-2k of code, depending
on the MCU instruction set and USB interfaces.
The bootloader on SAMD21 (which is 8k of code) implements \UF flashing with read-back capability (ie.,
a \UF file is exposed in the virtual FAT that contains current content of the 
MCU flash; as \MC embeds source code in binaries this lets the user drag the current \UF file from
a board into \MC browser window to load the project), legacy serial bootloader for Arduino, and a custom USB HID protocol
for flashing with a native application (but no drivers).


    \section{Evaluation}
\label{sec:evaluate}

This section evaluates the costs of the various layers of abstraction
in our platform used to create the simplified end-to-end experience. 
We evaluate the cost of these layers using a set of micro-benchmarks, written
in both C++ and Static TypeScript, for three microcontrollers: nrF52, SAMD21 (CPX device), and Atmega (Uno).
The C++ benchmarks isolate the performance
of \CO and provide a baseline, while the Static TypeScript benchmarks show the overhead
added by \MC. 

\subsection{C++ and Static TypeScript runtime}
The C++ of our platform breaks into two essential layers:
\begin{itemize}
\item \emph{\CON-core}: the \CO device runtime, against which we write C++ programs;
\item \emph{\MCN-common-packages}: the C++ and Static TypeScript wrapping \CON-core 
to make it known to \MC, against which we write Static TypeScript programs;
\end{itemize}

The CPX runtime is the largest, as the CPX has lots of on-board
components; it shares much of its code with other SAMD21 \MC targets.
The compiled C++ runtime is 114k in total, with \CO accounting for 29k, with an
additional 15k from libmbed. The \MCN-common-packages adds 20k, but also pulls in
49k of the math support libraries (floating point operations,
including trigonometry and number printing/parsing).
As the SAMD21 has 256k of flash, this code was not heavily optimized for space.

On UNO, which has 32k of flash, the much less complete \MC runtime is 8k, and \CO is 14k.

The TypeScript part of the runtime is 1060 statements on CPX and 216 on UNO.
In our samples 99\% of that runtime is tree-shaken away.

\subsection{Compiling Static TypeScript}
% data generated using 'node map-file-stats.js CIRCUIT_PLAYGROUND.map'
% maybe put it in a table?

When compiling the Static TypeScript benchmarks and runtime with code shaking we obtain 
a generated code density of 37.5 bytes per statement on CPX, 60.8 on AVR native, and 12.3 on AVR VM.
This excludes string literals (which don't change the picture significantly), but includes class 
vtables and number literals if any. Note that ARM and AVR native instructions are 2 bytes each,
while AVR VM instructions are between 0.5 and 3 bytes.

When compiling, the entire TypeScript program, including the runtime, is
passed to the TypeScript (TS) language service for parsing. Then, only the remaining
part of the program (after code shaking) is compiled to native code.
On a modern laptop, using Node.js, TS parsing and analysis takes about 0.1ms per statement,
while \MC compilation to native code takes about 1ms per statement.
While the TS compiler has been optimized for speed, 
\MCN's native compilation process hasn't been.
For example, for CPX the TS pass is dominated by compile the device runtime 
and takes about 100ms, whereas the \MC pass typically only includes a small user program
and a small bit of the runtime, resulting in less than 100ms. 
Thus, typical compilation times are in the range of 200ms for user programs
of ABC-XYZ lines. 

The AVR VM was specifically designed for high code density, since the C++ runtime
leaves less than 10k for TS runtime and user code on the Uno.
The interpreter is implemented in assembly and always included with the program and is around 0.5k.
There are about 30 opcodes, some of which can take 1 or 2 byte arguments. 
There are also a few combined opcodes, representing a sequence of one argument-less opcode,
and one with an argument, which improves code density by about 25\%.
Opcodes are direct offsets into the code of the interpreter, speeding up execution.
They operate on a stack (mainly for function calls) and a special scratch register.
There is essentially no stack space overhead compared to native AVR compilation.
The speed overhead is around 4x-5x (with respect to native) for computational tasks.


%\subsection{Implementation}
% •	\CO (SAMD21 and AVR): base runtime (C++ only)
% •	pxt
% •	pxt-common-packages: C++ and Static TypeScript
% •	pxt-adafruit
% •	pxt-arduino-uno
% •	pxt-monaco, pxt-blockly

\subsection{FLASH and RAM footprint}

\begin{itemize}

 % FLASH and RAM footprint
\item FLASH footprint of the combined runtime codebase split by role (\CON-core, \MCN-common-packages)
% Analyse MAP file. (Assuming PXT core is in build, else Michal to provide size of PXT core libs)?
\item RAM footprint of the combined runtime codebase, also split by role; 
% (libc, codal-core, pxt-common-packages, makecode). Analyse MAP file for static footprint. 
%Enable HEAP\_DEBUG, and run C++ blinky and PXT blinky.
% cost of async/fibers/handlers
\item Scaling of RAM footprint with number of active fibers (parallel recursive calls in TS. 
%      Should be predictable, but validates we scale linearly and sets some hard figures 
%      for a real device). C++ test.
\end{itemize}

\subsection{Time for common \CO operations}

\begin{itemize}
\item Context switch time (both in real terms (uS) and CPU instructions). 
% C++ test, flipping a GPIO or two duting context switch and measuring on scope.
\item Event handling time (both in real terms (uS) and CPU instructions. 
%      Both IMMEDIATE and THREADED mode). C++ test. flipping a GPIO or two during context 
%      switch sections and measuring on scope.
\item Async Procedure Call (APC) handling time (both in real terms (uS) and CPU instructions). 
% C++ test. flipping a GPIO or two during context switch and measuring on scope.
\end{itemize}




%mmoskal [10:10 AM] 
%`pxt checkdocs --snippets --re perf --stats`
% [10:11] 
% I compile empty sample first twice, to reduce JIT costs
% [10:12] 
% also, the first "compile prep" is slightly more costly, since it parses a hex file

\begin{itemize}
\item Time taken to compile and link simple program in browser (can we use some existing apps 
      here as case studies? E.g. Smiley emoji and fireflies?
% microbenchmarks
% in addition, Arduino "examples"
% instrument via node? or shell?
\item \UF File Size comparisons to HEX and bin. Time taken to complete a UF2 flash operation 
    (c.f. equivalent DAPLINK flash on micro:bit and avrdude flash on uno?)
\end{itemize}

\subsection{C++, Native vs Bytecode}
% - program benchmarks for C++, compiled MakeCode, interpreted MakeCode
%     - APP: A simple blinky
%     - Measures CPU time, RAM, Flash consumption
\begin{itemize}
\item Tight loop performance of C++, TS, and Blocks e.g. flipping a GPIO. PXT native compile.
\item Tight loop performance of C++, TS, and Blocks e.g. flipping a GPIO. PXT bytecode interpreted.
\end{itemize}

\subsection{Hardware resources}

% - efficient use of hardware resources, compared to Arduino, which uses spin loops and doesn't always use the hardware (bit bangs instead)


    \section{Related Work}
\label{sec:related}
%% potentially cut this
\subsection{Novice Programming Environments}

Arduino~\cite{buildingArduino2014} is an environment for programming microcontrollers, aimed at novices. However, its C++ based APIs introduces barriers for novice programmers~\cite{blikstein2013gears}. Scratch~\cite{ScratchCACM2009} is a widely adopted, event-based visual programming environment designed to introduce novice programmers to computer science concepts. Extensions enable the programming of physical devices with Scratch. However, devices require constant tethered connections to operate, restricting potential projects~\cite{dougherty2012maker}. ArduBlock~\cite{Ardubloc28:online} brings visual programming to the Arduino, but it lacks the event-based blocks Scratch users are familiar with.

With the environments above, additional software must be installed --- this creates barriers for novice users in restrictive environments. MakeCode and CODAL require \emph{no installation} to support a diverse user base and support \emph{event-based higher-level languages} to help beginners get a head start in the world of the microcontroller.

\subsection{Virtual Machine-based Languages}

Recently, virtual machines supporting most of the semantics of higher level languages like JavaScript, Java, and Python, have been ported to 32-bit microcontrollers by maker communities~\cite{dougherty2012maker}. Examples include: MicroPython~\cite{MicroPython}, CircuitPython, and Espruino~\cite{espruinoBook}. These VMs consume a large amount of RAM and flash memory, and run significantly slower than native languages.

The research community has worked to bring higher level languages to microcontrollers~\cite{koshy2005vmstar,st2009picobit,vaugon2015programming}. Rather than running a full-featured VM, others enable higher level languages to run efficiently by stripping out advanced language features, in favor of efficient, native execution~\cite{varma2004java}. Comparing these solutions to our solution is challenging due to a misalignment in evaluation metrics and microcontrollers. For example, the PICOBit uses an 8-bit MCU, and evaluates the cost of a VM, without the cost of a runtime environment. Simply accounting for a 32-bit MCU in this case, results in factor of 4 multiplication of most metrics.

Our approach bears most similarity to~\cite{varma2004java}, where we compile higher level languages to an \emph{optimized, event-driven} C++ runtime (\CON).

\subsection{Embedded Runtime Environments}

%Embedded runtime environments can range from simple C++ classes that control hardware, to real-time operating systems (RTOSs) with scheduling and memory management.

Arduino~\cite{buildingArduino2014} is an example of a simple platform where the developer uses high-level APIs to control hardware; there is no scheduler and memory management is discouraged, with a heavy emphasis on
the use of global variables.

TinyOS~\cite{levis2005tinyos}, Contiki~\cite{dunkels2012contiki}, RIOT OS~\cite{baccelli2013riot}, Mynewt~\cite{ApacheMy53:online}, mbed OS~\cite{ARMmbed}, and Zephyr~\cite{HomeZeph63:online} are RTOS solutions known widely in the systems community. The majority focus on the networking features of sensor based devices and commonly adopt a preemptive scheduling model, which leads to competition over resources resolved using locks and condition synchronization primitives. Contiki has a cooperative scheduler but uses proto-threads to store thread context --- local variables are not allowed as the context of the stack is not stored.

Although the platforms above are widely used by C/C++ developers, none of these existing solutions align well with the programming paradigms seen in higher level languages. \CO \emph{bridges the semantic gap between the higher level language and the microcontroller}, offering appropriate abstractions and higher level primitives written natively in C++.

\subsection{Flashing Microcontrollers}
There are two common ways to transfer a program to the flash of a microcontroller: for embedded developers, a specialized debugger chip; for hobbyists, a custom serial protocol~\cite{AVRDUDEA15:online}. Both approaches require operating system drivers. ARM's mbed platform provides DAPLink~\cite{GitHubAR5:online}, firmware that presents itself to an external computer as a USB pen drive. DAPLink exposes a virtual file system that caters for normal file system behavior and handles the decoding of Intel HEX files~\cite{IntelHEX} --- the firmware consumes 66 kB of flash and 13 kB RAM. \UF contributes a new file format that \emph{greatly simplifies} the virtual file system approach, reducing complexity of the firmware and code size.

% talk about old protocols for flashing protocols, talk about mbed, talk about UF2

% % * programming languages for microcontrollers
% %     - discuss low level languages
% %     - talk about approaches to higher level languages by communities
% %     - talk about approaches to higher level languages by research
% %     - contextualise against our approach
% \subsection{Programming languages for novice users}


% Java-through-C compilation: \cite{varma2004java}

% VMSTAR: synthesizing scalable runtime environments for sensor networks \cite{koshy2005vmstar}

% % * Embedded Runtime Environments
% %     -largely remains as is.
% \subsection{Embedded Runtime Environments}

% Typically written in C and/or C++, environments for MCU programming all share a common design goal: to support developers by providing primitives and programming abstractions. Platforms can range from simple C++ classes that control hardware, to real-time operating systems (RTOSs) with scheduling and memory management.

% Arduino~\cite{buildingArduino2014} is an example of a simple platform where the developer uses high-level APIs to control hardware; there is no scheduler, and memory management is discouraged through a heavy emphasis on global variables.

% TinyOS~\cite{levis2005tinyos}, Contiki~\cite{dunkels2012contiki}, RIOT OS~\cite{baccelli2013riot}, Mynewt~\cite{ApacheMy53:online} mbed OS~\cite{ARMmbed}, Zephyr~\cite{HomeZeph63:online} are RTOS solutions known widely in the embedded systems community. The majority focus on the networking features of sensor based devices and commonly adopt a preemptive scheduling model, which leads to competition over resources resolved using locks and condition synchronization primitives. Contiki has a cooperative scheduler, but uses proto-threads to store thread context --- local variables are not allowed as the context of the stack is not stored.

% Although the platforms above are widely used by C/C++ developers, none of these existing solutions align well with the programming paradigms seen in higher level languages. \CO bridges that semantic gap between the higher level language and the microcontroller, offering appropriate abstractions and higher level primitives written natively in C++.


% \subsection{Programming microcontrollers}

% \paragraph{Tethering a device} Scratch's device extensions~\cite{ScratchCACM2009} and Arduino's Firmata library~\cite{Firmata} take the tethered approach. However, in the educational and maker~\cite{dougherty2012maker} settings, MCUs often are embedded in projects. Here, this method has limited utility, as in such projects the MCU is powered by battery and is not connected to a host.

% \paragraph{Loading a pre-compiled native binary onto a device}
% Arduino~\cite{buildingArduino2014}, mbed~\cite{ARMmbed}, and other C-based solutions require installation of additional software and knowledge of low-level languages. These requirements do not suit a diverse demographic of inexperienced users.

% \paragraph{Interpretting pre-compiled bytecode on the device}
% Java for embedded systems~\cite{ClausenTOPLAS}

% suffer in terms of performance on small MCUs, operating slower than native C++ code, and consuming most of the resources (flash and RAM) of the MCU.

% \paragraph{Loading a compiler and virtual machine onto a device}
% MicroPython~\cite{MicroPython} and Espruino~\cite{espruinoBook};

% Our platform slots into options 2 and 3, using the \MC architecture and USB pen drives to avoid the need for installation of native applications or device drivers. Our use of STS provides better performance than the MicroPython and Espruino solution, as demonstrated in the evaluation,
% and allows us to fit in the very constrained space of the Uno.




% \subsection{Programming microcontrollers}

% talk about old protocols for flashing protocols, talk about mbed, talk about UF2

    \section{Conclusion}
\label{sec:conclude}

We have presented \MCN: a \emph{no installation}, web-based programming environment, that supports novice programmers with \emph{block-based and text-based higher-level languages}, and compiles programs \emph{in the browser}. So as to not compromise the spatial efficiency of the microcontroller, we created \CON: a C++ runtime that \emph{bridges the semantic gap} between higher level languages in \MC and C++. To transfer programs compiled by \MC to the microcontroller without the installation of any drivers, we created \UFN: a new bootloader and file format that enables the \emph{simplified}, \emph{driverless} programming of microcontrollers.

Combined, our approach to running higher level languages on microcontrollers is up to 50x more performant compared to other approaches. Further, by using modern tooling, and higher level languages, our approach lowers the barrier to entry for microcontroller programming.

% We have presented and evaluated a new platform designed to bring modern language features and tooling to microcontrollers. Our aim was to do this in an extensible way which supports novice programmers with block-based programming while providing a progression path to a text-based scripting language and ultimately to C++. Our platform includes a new C++ runtime called \CO which is designed to make efficient use of the limited resources on a microcontroller. A statically-typed subset of TypeScript forms the basis for both blocks- and text-based programs, created and compiled using a new web-based IDE named \MC.

% Our aspiration is to enable a new paradigm for programming pretty-much anything, even an Arduino Uno-class MCU, by anyone -- novices and professional developers alike, from anywhere, i.e. without the need for traditional heavyweight embedded toolchains and IDEs. Our open-source implementation is in daily use, with thousands of users writing programs targeted at several different MCU-based devices. In this sense, we have achieved our goal. We also have anecdotal evidence that our platform -- in terms of both language and tooling -- is intuitive to professional developers with no experience of embedded development.


    %% Bibliography
    \bibliography{paper}

    % \appendix
\pagebreak
\section{Static TypeScript Subtype Relation}


In STS, $S$ is a subtype of a type $T$ if one of the following is true:
\begin{itemize}
\item $S$ and $T$ are identical types;
\item $S$ is the Undefined type;
\item $S$ is the Null type and $T$ is not the Undefined type;
\item $S$ is an enum type and $T$ is the primitive type Number;
\item $S$ is a string literal type and $T$ is the primitive type String;
\item $S$ and $T$ are class types and all the following are true:
\begin{itemize}
  \item $S$ is derived from $T$ (via \emph{extends} clauses);
  \item checkProps($S$,$T$) holds;
\end{itemize}
\item $S$ is a class/record type and $T$ is a record type and
\begin{itemize}
  \item checkProps($S$,$T$) holds;
\end{itemize}
\item $S$ and $T$ are function types such that all the following hold:
\begin{itemize}
  \item $S$ has at least as many parameters as $T$;
  \item each parameter type in $T$ is a subtype of the corresponding parameter type in $S$;
  \item the result type of $T$ is Void, or the result type of $S$ is a subtype of that of $T$;
\end{itemize}
\item $S$ and $T$ are array types and all the following hold:
\begin{itemize}
\item $T$ has a numeric index signature with element type $U$,
    and $S$ has a numeric index signature with element type $V$
    such that $V$ is a subtype of $U$;
\item checkProps($S$,$T$) holds.
\end{itemize}
\end{itemize}

Given types $S$ and $T$, checkProps($S$,$T$) holds if for each property $N$ in $T$,
$S$ has a property $M$ where all of the following are true:
\begin{itemize}
\item $M$ and $N$ have the same name;
\item the type of $M$ is a subtype of the type of $N$;
\item $M$ and $N$ are both public, or $M$ and $N$ are both
      private (protected) and originate in the same declaration,
      or $N$ is protected and $S$ is a class derived from class $T$
\end{itemize}

\section{Artifact appendix}

Submission and reviewing guidelines and methodology: \\
{\em http://cTuning.org/ae/submission.html}

%%%%%%%%%%%%%%%%%%%%%%%%%%%%%%%%%%%%%%%%%%%%%%%%%%%%%%%%%%%%%%%%%%%%%
\subsection{Abstract}

This artifact allows others to reproduce the results seen in this paper for MakeCode and Codal, using the BBC micro:bit. The artifact contains an offline build environment for codal and MakeCode, allowing evaluators to test and build programs locally. In addition, we also provide espruino and micropython virtual machines to further increase repeatability of our results. Evaluators should download the virtual machine containing all pre-requisite tools, and use an oscilloscope to observe wave forms (used for timing) generated by the micro:bit, and a serial terminal to observe results reported from the micro:bit over serial.


\subsection{Artifact check-list (meta-information)}

{\small
\begin{itemize}
  \item {\bf Program:} MakeCode \& Codal
  \item {\bf Compilation:} arm-none-eabi-gcc
  \item {\bf Binary:} espruino, and micropython binaries included; others compiled during testing
  \item {\bf Run-time environment:} Codal
  \item {\bf Hardware:} BBC micro:bit
  \item {\bf Output:} Waveforms, and serial output
  \item {\bf Publicly available?:} Yes
  \item {\bf Artifacts publicly available?:} Yes
  \item {\bf Artifacts functional?:} Yes
  \item {\bf Artifacts reusable?:} Yes
  \item {\bf Results validated?:} Yes
\end{itemize}

%%%%%%%%%%%%%%%%%%%%%%%%%%%%%%%%%%%%%%%%%%%%%%%%%%%%%%%%%%%%%%%%%%%%%
\subsection{Description}

\subsubsection{How delivered}

The artifact is available on GitHub:\\[5pt]\url{https://lancaster-university.github.io/lctes-artefact-evaluation/}\\[5pt] And a virtual machine, based on debian, containing all the required software to reproduce our results is available here:\\[5pt]\url{https://drive.google.com/open?id=1nxiorz6NRqjen89G59RCOEMklqAyaUv7}

\subsubsection{Hardware dependencies}

\begin{itemize}
    \item A BBC micro:bit
    \item An oscilloscope
    \item A computer capable of running a virtual machine
\end{itemize}

\subsubsection{Software dependencies}

\begin{itemize}
    \item A virtual machine obtained from the URL above.
    \item A serial terminal.

\end{itemize}

%%%%%%%%%%%%%%%%%%%%%%%%%%%%%%%%%%%%%%%%%%%%%%%%%%%%%%%%%%%%%%%%%%%%%
\subsection{Installation}

Use virtual box to install the image located at:\\[5pt]\url{https://drive.google.com/open?id=1nxiorz6NRqjen89G59RCOEMklqAyaUv7}\\[5pt]
and the VirtualBox extension pack:\\[5pt]\url{https://www.virtualbox.org/wiki/Downloads}
%%%%%%%%%%%%%%%%%%%%%%%%%%%%%%%%%%%%%%%%%%%%%%%%%%%%%%%%%%%%%%%%%%%%%
\subsection{Experiment workflow}

Tests generally follow the following sequence of steps:

\begin{enumerate}
    \item Perform small program modifications.
    \item Compile the program.
    \item Transfer program to the micro:bit (flashing).
    \item Observe either a waveform generated by the micro:bit using an oscilloscope, or serial output from the micro:bit using a serial program.
\end{enumerate}

%%%%%%%%%%%%%%%%%%%%%%%%%%%%%%%%%%%%%%%%%%%%%%%%%%%%%%%%%%%%%%%%%%%%%
\subsection{Evaluation and expected result}

We expect the results to be the same as those reported in the paper. The observed waveforms may differ in time due to different compilers, oscilloscopes, and oscilloscope calibration.

%%%%%%%%%%%%%%%%%%%%%%%%%%%%%%%%%%%%%%%%%%%%%%%%%%%%%%%%%%%%%%%%%%%%%
\subsection{Experiment customization}

All tests provided have a clear set of corresponding instructions that evaluators should follow to observe the same results. Any steps involving customisation have been minimised.

%%%%%%%%%%%%%%%%%%%%%%%%%%%%%%%%%%%%%%%%%%%%%%%%%%%%%%%%%%%%%%%%%%%%%
\subsection{Notes}

The virtual machine contains a folder named `evaluators' which is placed in the home directory of the lctes user. The username for the virtual machine is: \textit{lctes} and the password is: \textit{lctes2018}. To become super user, type \textit{su} in a terminal, and enter the same password (\textit{lctes2018}).

Once logged in, and in the `evaluators' directory, you can view the tests as markdown files in the `docs' directory. Alternately, these markdown documents can also be viewed on the web by running `mkdocs serve' in the evaluators folder, or browsing to:\\[5pt]\url{https://lancaster-university.github.io/lctes-artefact-evaluation/}\\[5pt] Which is a pre-built, and hosted version produced from the same source.

We recommend that you add the micro:bit usb device using the machine settings tab in virtual box as shown in the image below:\\

\includegraphics[width=\columnwidth]{images/virtualbox.png}

We also have a convenience script for mounting a shared folder between the host and the vm. Simply create a shared folder named `lctes-vm-dir' and run `sh mount.sh' (contained in evaluators) as a super user to mount the shared folder to vb-share (also contained in evaluators). Shared folder creation in VirtualBox is pictured below:\\

\includegraphics[width=\columnwidth]{images/shared-folder.png}

    \end{document}
